\documentclass[11pt]{article}
\usepackage[top=1.5cm,bottom=1.5cm]{geometry}
\usepackage[T1]{fontenc}
\usepackage{url}
\usepackage{hyperref}
\usepackage{xcolor}
% 
\usepackage{amsthm}
\usepackage{amsmath}
\usepackage{amssymb}
\usepackage{ifthen}
\usepackage{mathpartir}
\usepackage{mathtools}
\definecolor{pblue}{rgb}{0.13,0.13,1}
\definecolor{pgreen}{rgb}{0,0.5,0}
\definecolor{pred}{rgb}{0.9,0,0}
\definecolor{pgrey}{rgb}{0.46,0.45,0.48}
\definecolor{beige}{rgb}{0.96, 0.96, 0.86}
\definecolor{cosmiclatte}{rgb}{1.0, 0.97, 0.91}
\definecolor{cream}{rgb}{1.0, 0.99, 0.82}
\definecolor{ggreen}{rgb}{0.0, 0.5, 0.0}
\definecolor{rred}{rgb}{1.0, 0.13, 0.32}
\definecolor{bblue}{rgb}{0.0, 0.5, 1.0}
\definecolor{upmaroon}{rgb}{0.0, 0.2, 0.67}

\renewcommand{\emph}[1]{\textbf{#1}}

\DeclareMathOperator{\defi}{\triangleq}

\def\signed #1{{\leavevmode\unskip\nobreak\hfil\penalty50\hskip2em
  \hbox{}\nobreak\hfil(#1)%
  \parfillskip=0pt \finalhyphendemerits=0 \endgraf}}

\newsavebox\mybox
\newenvironment{aquote}[1]
  {\savebox\mybox{#1}\begin{quote}}
  {\signed{\usebox\mybox}\end{quote}}

% \mode<handout>{\newcommand\handoutmcd[1]{\begin{comment} #1 \end{comment} }}

\newcommand{\separator}[1]{
  \begin{frame}
    \begin{center}
    \Huge{#1}
  \end{center}
\end{frame}
}


\newcommand\blfootnote[1]{%
  \begingroup
  \renewcommand\thefootnote{}\footnote{#1}%
  \addtocounter{footnote}{-1}%
  \endgroup
}

\DeclareMathOperator{\concat}{\text{\texttt{++}}}

\newcommand{\menquote}[1]{{\text{``} #1 \text{''}}} 

\DeclareMathOperator{\meq}{=}
\DeclareMathOperator{\aeq}{=_{\alpha}}
\DeclareMathOperator{\naeq}{\neq_{\alpha}}



\newcommand{\kw}[1]{\mathtt{#1}} 
\newcommand{\var}[1]{{\color{bblue}#1}}
\newcommand{\tvar}[1]{{\color{upmaroon}#1}}

\newcommand{\naturals}{\mathbb{N}}
\newcommand{\bnfsep}{\; | \;}
\newcommand{\qst}{\, : \,}
\newcommand{\msubs}[3]{#1[#2 /{#3}]}
\newcommand{\twosubs}[5]{#1[#2 /{\var{#3}}, #4 / {\var{#5}}]}
\newcommand{\subs}[3]{\msubs{#1}{#2}{\var{#3}}}
\newcommand{\tsubs}[3]{\msubs{#1}{#2}{\tvar{#3}}}


\newcommand{\dom}[1]{\mathit{dom}(#1)}
\newcommand{\QUAL}{{\mathsf{Qua}}}
\newcommand{\TYPES}{\tvar{\mathsf{Typ}}}
\newcommand{\PTYPES}{{\mathsf{PreTyp}}}
\newcommand{\EXPS}{\var{\mathsf{Exp}}}
\newcommand{\AEXPS}{\mathsf{AExp}}
\newcommand{\BEXPS}{\mathsf{BExp}}
\newcommand{\STMTS}{\mathsf{Stmt}}
\newcommand{\VARS}{\mathcal{V}}


\newcommand{\lgge}[1]{\textbf{#1}}
%
\newcommand{\lggeE}{\lgge{E}}
\newcommand{\lggeEF}{\lgge{EF}}
\newcommand{\lggeT}{\lgge{T}}
\newcommand{\lggeW}{\lgge{W}}
\newcommand{\lggeF}{\lgge{F}}
\newcommand{\lggeL}{\lgge{L}}
\newcommand{\lggeX}{\lgge{EX}}
%



\newcommand{\binop}[3]{\kw{#1}(#2; #3)}
\newcommand{\unop}[2]{\kw{#1}(#2)}


\newcommand{\litstr}[1]{\menquote{#1}}

\newcommand{\enum}[1]{\kw{num}\ifthenelse{\equal{#1}{}}{}{[#1]}}
\newcommand{\estr}[1]{\kw{str}\ifthenelse{\equal{#1}{}}{}{[#1]}}
\newcommand{\ebool}[1]{\kw{bool}\ifthenelse{\equal{#1}{}}{}{[#1]}}
\newcommand{\elbool}[2]{\kw{bool}_{#2}\ifthenelse{\equal{#1}{}}{}{[#1]}}
\newcommand{\ktrue}{\kw{true}}
\newcommand{\kfalse}{\kw{false}}
\newcommand{\eand}[2]{\binop{and}{#1}{#2}}


\newcommand{\eplus}[2]{\binop{plus}{#1}{#2}}
\newcommand{\eequal}[2]{\binop{equal}{#1}{#2}}
\newcommand{\etimes}[2]{\binop{times}{#1}{#2}}
\newcommand{\ecat}[2]{\binop{cat}{#1}{#2}}
\newcommand{\elen}[1]{\unop{len}{#1}}
\newcommand{\elet}[3]{\kw{let}(#1; \var{#2} . \ #3)}
\newcommand{\ediv}[2]{\binop{div}{#1}{#2}}
\newcommand{\eerror}{\kw{{\color{rred} error}}}
\newcommand{\eshow}[1]{\kw{show}(#1)}

\newcommand{\klet}{\kw{let}}
\newcommand{\kbe}{\kw{be}}
\newcommand{\kin}{\kw{in}}
\newcommand{\clet}[3]{\klet \ \var{#2} \  \kbe \ #1  \ \kin \  #3}
\newcommand{\cleta}[2]{\klet \ \var{#2} \  \kbe \ #1}
\newcommand{\cletb}[1]{\kin \  #1}



\newcommand{\grulename}[1]{\textsc{#1}}
\newcommand{\tyrule}[3]{\inferrule*[Left=#1]{#2}{#3}}
\newcommand{\tyrulename}[2]{#1-ty-#2}
\newcommand{\etyrulename}[1]{\tyrulename{\lggeE}{{#1}}}
\newcommand{\eftyrulename}[1]{\tyrulename{\lggeEF}{{#1}}}
\newcommand{\ttyrulename}[1]{\tyrulename{\lggeT}{{#1}}}
\newcommand{\ftyrulename}[1]{\tyrulename{\lggeF}{{#1}}}
\newcommand{\ltyrulename}[1]{\tyrulename{\lggeL}{{#1}}}


\newcommand{\tyjudge}[3]{#1 \vdash #2 : #3}


\newcommand{\valjudge}[1]{#1 \ \mathsf{{\color{ggreen} val}}}
\newcommand{\errjudge}[1]{#1 \ \mathsf{{\color{rred} err}}}


\newcommand{\semrule}[3]{\inferrule*[Left=#1]{#2}{#3}}
\newcommand{\semrulename}[2]{#1-se-#2}
\newcommand{\esemrulename}[1]{\semrulename{\lggeE}{{#1}}}
\newcommand{\efsemrulename}[1]{\semrulename{\lggeEF}{{#1}}}
\newcommand{\tsemrulename}[1]{\semrulename{\lggeT}{{#1}}}
\newcommand{\wsemrulename}[1]{\semrulename{\lggeW}{{#1}}}
\newcommand{\fsemrulename}[1]{\semrulename{\lggeF}{{#1}}}
\newcommand{\lsemrulename}[1]{\semrulename{\lggeL}{{#1}}}

\newcommand{\trans}{\longmapsto}
\newcommand{\transC}{\longmapsto^{\ast}}
\newcommand{\jtrans}[2]{#1 \trans #2}
\newcommand{\jtransC}[2]{#1 \transC #2}



\newcommand{\anorule}[2]{\inferrule*{#1}{#2}}

%%%%%% LANGUAGE EF %%%%%%%%%%%%%%

\newcommand{\tyarr}[2]{\kw{arr}{(#1; #2)}}
\newcommand{\tylist}[1]{\kw{list}{(#1)}}
\newcommand{\ctyarr}[2]{#1 \rightarrow #2}
\newcommand{\elam}[3]{\kw{lam}{\{#1\}}{(\var{#2}. \ #3)}}
\newcommand{\clam}[3]{\lambda(\var{#2}:#1) . \ (#3)}
\newcommand{\eapp}[2]{\kw{ap}(#1; #2)}
\newcommand{\capp}[2]{(#1) \ (#2)}


%%%%%% LANGUAGE T %%%%%%%%%%%%%%

\newcommand{\enot}[1]{\kw{not}(#1)}
\newcommand{\ezero}[1]{\kw{isZero}(#1)}

\newcommand{\tynat}{\kw{nat}}
\newcommand{\ez}{\kw{z}}
\newcommand{\esucc}[1]{\kw{s}(#1)}
\newcommand{\erec}[5]{\kw{rec}\{#1; \var{#2} . \var{#3} . #4\} (#5)}
\newcommand{\crec}[5]{\kw{rec} \ #5 \ \{ \ez \hookrightarrow  #1
  \ | \
  \esucc{\var{#2}} \ \kw{with} \ \var{#3} \hookrightarrow #4 \}}


\newcommand{\eiter}[4]{\kw{iter}\{ #1; \var{#2} . #3 \} (#4)}
\newcommand{\citer}[4]{\kw{iter} \ #4 \ \{ \ez \hookrightarrow  #1
  \ | \
  {\var{#2}} \hookrightarrow #3 \}}


\newcommand{\edeclist}[5]{\kw{case}\{ #1; (\var{#2}, \var{#3}). #4 \} (#5)}
\newcommand{\cdeclist}[5]{\kw{case} \ #5 \ \{ [ \, ] \hookrightarrow  #1
  \ | \
  {(\var{#2}::\var{#3})} \hookrightarrow #4 \}}

\newcommand{\econs}[2]{\kw{cons}(#1;#2)}
\newcommand{\ccons}[2]{(#1::#2)}




%%%%%%%%%%%%%%%% DATA TYPES

\newcommand{\tyunit}{\kw{unit}}
\newcommand{\typrod}[2]{\kw{prod}(#1;#2)}
\newcommand{\etriv}{\kw{triv}}
\newcommand{\ctriv}{\langle \rangle}
\newcommand{\epair}[2]{\binop{pair}{#1}{#2}}
\newcommand{\cpair}[2]{\langle #1, #2 \rangle}
\newcommand{\eprl}[1]{\kw{pl}(#1)}
\newcommand{\eprr}[1]{\kw{pr}(#1)}
\newcommand{\cprl}[1]{#1 \cdot \kw{l}}
\newcommand{\cprr}[1]{#1 \cdot \kw{r}}



\newcommand{\tyvoid}{\kw{void}}
\newcommand{\tysum}[2]{\kw{sum}(#1;#2)}
\newcommand{\eabort}[2]{\kw{abort}\{#1\}(#2)}
\newcommand{\cabort}[1]{\kw{abort}(#1)}
\newcommand{\einl}[3]{\kw{inl}\{#1;#2\}(#3)}
\newcommand{\einr}[3]{\kw{inr}\{#1;#2\}(#3)}
\newcommand{\cinl}[1]{\kw{l} \cdot #1}
\newcommand{\cinr}[1]{\kw{r} \cdot #1}
\newcommand{\ecase}[5]{\kw{case}(#1; \var{#2} . \ #3 ; \var{#4} . \ #5)}
\newcommand{\ccase}[5]{\kw{case}\ #1 \ \{\kw{l} \cdot \var{#2}  \hookrightarrow #3 \ | \ 
  \kw{r} \cdot \var{#4} \hookrightarrow #5\}}


%%%%%%%%%%%%%%%%%% T LANGUAGE
\newcommand{\bnum}[1]{\overline{#1}}


%%%%%%%%%%%%%%%%%% F LANGUAGE
\newcommand{\tyall}[2]{\kw{all}{(\tvar{#1} . #2)}}
\newcommand{\ctyall}[2]{\forall(\tvar{#1} . #2)}

\newcommand{\eLAM}[2]{\kw{Lam}(\tvar{#1}.#2)}
\newcommand{\cLAM}[2]{\Lambda (\tvar{#1}) . #2}
\newcommand{\eAPP}[2]{\kw{Ap}\{#1\}(#2)}
\newcommand{\cAPP}[2]{#2 \lceil #1 \rceil}


\newcommand{\typeok}{\mathsf{{\color{ggreen} type}}}
\newcommand{\typejudge}[2]{#1 \vdash #2 \ \typeok}

\DeclareMathOperator{\ctxtsep}{;}


\newcommand{\tyFjudge}[4]{#1 \ctxtsep #2 \vdash #3 : #4}


%%%%%%%%%%%%%%%%%%%%%%% L LANGUAGE

\newcommand{\ellam}[4]{\kw{llam}_{#4}{\{#1\}}{(\var{#2}.#3)}}
\newcommand{\cllam}[4]{\lambda_{#4} (\var{#2}:#1) . #3}
\newcommand{\elpair}[3]{\kw{pair}_{#3}({#1; #2)}}
\newcommand{\clpair}[3]{#3 \ \langle #1, #2 \rangle}

\newcommand{\esplit}[4]{\kw{let}(#1; (\var{#2}, \var{#3}) . #4)}

\newcommand{\ksplit}{\kw{let}}
\newcommand{\kas}{\kw{as}}
\newcommand{\csplit}[4]{\ksplit \  \var{#2}, \var{#3} \ \kbe \ #1 \  \kin \  #4}

\newcommand{\unq}{{\color{ggreen}\mathsf{un}}}
\newcommand{\linq}{{\color{rred}\mathsf{lin}}}
\newcommand{\ptype}[2]{#1 \ #2}


\DeclareMathOperator{\ctxtsplit}{\circ}


\newcommand{\unres}[1]{\unq(#1)}
\newcommand{\linear}[1]{\linq(#1)}


%%%%%%%%%%% WHILE LANGUAGE

\newcommand{\eleq}[2]{\binop{leq}{#1}{#2}}

\newcommand{\sskip}{\kw{skip}}
\newcommand{\csseq}[2]{#1; #2}
\newcommand{\sseq}[2]{\binop{seq}{#1}{#2}}
\newcommand{\site}[3]{\kw{if}(#1; #2 ;#3)}
\newcommand{\csite}[3]{\kw{if}\ #1 \ \kw{then} \ #2 \ \kw{else} \ #3}
\newcommand{\swhile}[2]{\binop{while}{#1}{#2}}
\newcommand{\cwhile}[2]{\kw{while} \ #1 \ \kw{do} \ #2}

\newcommand{\sset}[2]{\kw{set}(\var{#1};#2)}
\newcommand{\cset}[2]{\var{#1} \leftarrow #2}


\newcommand{\upmap}[3]{#1[\var{#2} \mapsto #3]}

\newcommand{\wtyrulename}[1]{\tyrulename{\lggeW}{{#1}}}
\newcommand{\wconf}[2]{\langle #1, #2 \rangle}


\newcommand{\wejudge}[3]{#1 \vdash #2 : #3}
\newcommand{\wsjudge}[2]{#1 \vdash #2 \ {\color{ggreen} \mathsf{ok}}}


\newcommand{\bsjudge}[2]{#1 \Downarrow #2}


%%%%%%%%%%%%% UNTYPED LAMBDA

\newcommand{\eulam}[2]{\kw{lam}{(\var{#1}.#2)}}
\newcommand{\culam}[2]{\lambda \var{#1} . #2}


%%% Local Variables:
%%% mode: latex
%%% TeX-master: "main"
%%% End:

% 

\newcommand{\HL}[1]{\boxed{#1}}


\begin{document}

\title{CO663 Part II: Worksheet 4 (\emph{assessed})}

\date{\vspace{-15ex}} 
\maketitle

% \pagenumbering{gobble}

\newcommand{\lazyeval}[1]{\framebox{\parbox[c][#1]{\textwidth}{
      \color{white}{h}% 
    }}}


\section{Introduction}

The objective of this class is to put in practice the concepts of
programming language design by adding \emph{lists} to language
$\lggeT$.
% 
This is an individual exercise.

\begin{center}
  \emph{This class assessment is worth {\color{red} 3} marks.}
\end{center}

Submit your work via \emph{Moodle} (within the deadline specified on the
Moodle page).


\section{Syntax} 

Here is the syntax for $\lggeT$ with lists (new parts of \HL{highlighted}).

\[
\begin{array}{llclll}
  \TYPES & \tau & \Coloneqq & \tynat  & \tynat & \text{natural}
  \\
         &&& \HL{\tylist{\tau}} &  \HL{ [\tau] } & \text{lists}
  \\
         &&& \tyarr{\tau_1}{\tau_2}  & \tau_1 \rightarrow \tau_2 & \text{function}
  \\
  \\ 
  \EXPS & e & \Coloneqq & \var{x} & \var{x} & \text{variable}
  \\
         &&& \ez &  \ez & \text{zero}
  \\
         &&& \esucc{e} &  \esucc{e} & \text{successor}
  \\
         &&& \elam{\tau}{x}{e} & \clam{\tau}{x}{e} & \text{abstraction}
  \\
         &&& \eapp{e_1}{e_2} & \capp{e_1}{e_2} & \text{application}
  \\
         &&& \eiter{e_0}{x}{e_1}{e} & {\citer{e_0}{x}{e_1}{e}}   & \text{iterator}
  \\
         &&& \HL{[\,]} & \HL{[\,]} & \text{empty list}
  \\
         &&& \HL{\econs{e}{e'}} & \HL{\ccons{e}{e'}} & \text{list constructor}
  \\                                   
         &&& \HL{\edeclist{e_0}{x}{y}{e_1}{e}} & \HL{{\cdeclist{e_0}{x}{y}{e_1}{e}}}   &\text{list deconstructor}
\end{array}
\]

The construct $\econs{e}{e'}$ simply builds a new list from an element
$e$ and another list $e'$ by adding $e$ at the front of $e'$.
%
The construct $\edeclist{e_0}{x}{y}{e_1}{e}$ is a
\emph{binder} which introduces two new variables ($\var{x}$ and
$\var{y}$).
%
This construct should behave differently when $e$ is the empty list
(behave as $e_0$). If $e$ is non-empty, then it is divided into a head
(bound to $\var{x}$) and a tail (bound to $\var{y}$).

\section{Exercise 1: add typing rules for the new syntax constructs}
Here are the typing rules for $\lggeT$, add new rules to support lists.
\[
\tyrule
{\ttyrulename{var}}
{\,}
{\tyjudge{\Gamma_1, \var{x}: \tau, \Gamma_2}{\var{x}}{\tau} }
\qquad\qquad\qquad
\tyrule
{\ttyrulename{nat}}
{\,}
{\tyjudge{\Gamma}{\ez}{\tynat}}
\qquad\qquad\qquad
\tyrule
{\ttyrulename{num}}
{\tyjudge{\Gamma}{{e}}{\tynat{}}}
{\tyjudge{\Gamma}{\esucc{e}}{\tynat{}}}
\]
% 
\[
\tyrule
{\ttyrulename{lam}}
{\tyjudge{\Gamma, \var{x} : \tau_1}{e}{\tau_2}}
{\tyjudge{\Gamma}{\elam{\tau_1}{x}{e}}{\tyarr{\tau_1}{\tau_2}}}
\qquad\qquad\qquad
\tyrule
{\ttyrulename{ap}}
{
  \tyjudge{\Gamma}{e_1}{\tyarr{\tau_1}{\tau}}
  \and
  \tyjudge{\Gamma}{e_2}{\tau_1}
}
{
  \tyjudge{\Gamma}{\eapp{e_1}{e_2}}{\tau}
}
\]
\[
\tyrule
{\ttyrulename{ite}}
{
  \tyjudge{\Gamma}{e}{\tynat}
  \and
  \tyjudge{\Gamma}{e_0}{\tau}
  \and
  \tyjudge{\Gamma, \var{x} : \tau}{e_1}{\tau}
}
{
  \tyjudge{\Gamma}{\eiter{e_0}{x}{e_1}{e}}{\tau}
}
\]


\section{Exercise 2: add semantic rules for the new syntax constructs}


Here are the evaluation rules for $\lggeT$, add new rules to support lists.

% \subsection{Values}
\[
\semrule{\tsemrulename{z}}
{\,}
{\valjudge{\ez}}
\qquad \qquad  \qquad
\semrule{\tsemrulename{s}}
{\valjudge{e}}
{\valjudge{\esucc{e}}}
\qquad \qquad  \qquad
\semrule
{\tsemrulename{lam}}
{\,}
{\valjudge{\elam{\tau}{x}{e}}}
\]



% \subsection{Reductions}

\[
\semrule
{\tsemrulename{ss}}
{\jtrans{e}{e'}}
{
  \jtrans{\esucc{e}}{\esucc{e'}}
}
\qquad \qquad  \qquad
% 
\semrule
{\tsemrulename{ap}}
{\jtrans{e_1}{e'_1}}
{
  \jtrans{\eapp{e_1}{e_2}}{\eapp{e'_1}{e_2}}
}
\]
% 
% 
\[
\semrule
{\tsemrulename{lan}}
{\valjudge{e_1} 
  \and \jtrans{e_2}{e'_2}
}
{\jtrans{\eapp{e_1}{e_2}}{\eapp{e_1}{e'_2}}}
\qquad \qquad  \qquad
\semrule
{\tsemrulename{lav}}
{\valjudge{e_2}}
{\jtrans{\eapp{\elam{\tau}{x}{e_1}}{e_2}}{\subs{e_1}{e_2}{x}}}
\]
% 
\[
\semrule
{\tsemrulename{rin}}
{\jtrans{e}{e'}}
{
  \jtrans{\eiter{e_0}{x}{e_1}{e}}{\eiter{e_0}{x}{e_1}{e'}}
}
\qquad \qquad  \qquad
% 
\semrule
{\tsemrulename{r0}}
{\,}
{
  \jtrans{\eiter{e_0}{x}{e_1}{\ez}}{e_0}
}
% 
\]
\[
%  
\semrule
{\tsemrulename{rs}} 
{\valjudge{{e}}}
{
  \jtrans{\eiter{e_0}{x}{e_1}{\esucc{e}}}{\subs{e_1}{{\eiter{e_0}{x}{e_1}{{e}}}}{x}}
}
\]


\section*{\LaTeX\ template}
The \LaTeX\ sources of this document are available in:
\url{https://github.com/ukc-co663/pl-design-latex-template}.

% \end{document}

\section{Model answer}

\subsection{Typing rules}
\[
\tyrule
{\ttyrulename{dlist}}
{
  \tyjudge{\Gamma}{e}{\tylist{\tau}}
  \and
  \tyjudge{\Gamma}{e_0}{\tau'}
  \and
  \tyjudge{\Gamma, \var{x} : \tau, \var{y}: {\tylist{\tau}}}{e_1}{\tau'}
}
{
  \tyjudge{\Gamma}{\edeclist{e_0}{x}{y}{e_1}{e}}{\tau'}
}
\]



\[
\tyrule
{\ttyrulename{e}}
{\,}
{
  \tyjudge{\Gamma}{[\,]}{\tylist{\tau}}
}
\qquad \qquad  \qquad \qquad
\tyrule
{\ttyrulename{cons}}
{
  \tyjudge{\Gamma}{e}{{\tau}}
  \and
  \tyjudge{\Gamma}{e'}{\tylist{\tau}}
}
{
  \tyjudge{\Gamma}{\econs{e}{e'}}{\tylist{\tau}}
}
\]


\subsection{Semantic rules}

\[
\semrule{\tsemrulename{cons}}
{\valjudge{e}
  \and 
  \valjudge{e'}
}
{\valjudge{\econs{e}{e'}}}
\qquad \qquad  \qquad \qquad
\semrule{\tsemrulename{empty}}
{\,
}
{\valjudge{[\,]}}
\]

\[
\semrule
{\tsemrulename{cin}}
{\jtrans{e}{e'}}
{
  \jtrans{\edeclist{e_0}{x}{y}{e_1}{e}}{\edeclist{e_0}{x}{y}{e_1}{e'}}
}
\qquad \qquad  \qquad
% 
\semrule
{\tsemrulename{c0}}
{\,}
{ 
  \jtrans{\edeclist{e_0}{x}{y}{e_1}{[\,]}}{e_0}
}
% 
\]
\[
%  
\semrule
{\tsemrulename{cs}}
{\,}
{
  \jtrans{\edeclist{e_0}{x}{y}{e_1}{\econs{e}{e'}}}
  {\twosubs
    {e_1}
    {e}{x}
    {e'}{y}
  }
}
\]


\end{document}


%%% Local Variables:
%%% mode: latex
%%% TeX-master: t
%%% End:
