\documentclass[11pt]{article}
\usepackage[top=1.5cm,bottom=1.5cm]{geometry}
\usepackage[T1]{fontenc}
\usepackage{url}
\usepackage{hyperref}
\usepackage{xcolor}
% 
\usepackage{amsthm}
\usepackage{amsmath}
\usepackage{amssymb}
\usepackage{ifthen}
\usepackage{mathpartir}
\usepackage{mathtools}
\input{../../slides/macros}
% 

\newcommand{\HL}[1]{\boxed{#1}}


\begin{document}

\title{CO663 Part II: Worksheet 4 (\emph{assessed})}

\date{\vspace{-15ex}} 
\maketitle

% \pagenumbering{gobble}

\newcommand{\lazyeval}[1]{\framebox{\parbox[c][#1]{\textwidth}{
      \color{white}{h}% 
    }}}


\section{Introduction}

The objective of this class is to put in practice the concepts of
programming language design by adding \emph{lists} to language
$\lggeT$.
% 
This is an individual exercise.

\begin{center}
  \emph{This class assessment is worth {\color{red} 3} marks.}
\end{center}

Submit your work via \emph{Moodle} (within the deadline specified on the
Moodle page).


\section{Syntax} 

Here is the syntax for $\lggeT$ with lists (new parts of \HL{highlighted}).

\[
\begin{array}{llclll}
  \TYPES & \tau & \Coloneqq & \tynat  & \tynat & \text{natural}
  \\
         &&& \HL{\tylist{\tau}} &  \HL{ [\tau] } & \text{lists}
  \\
         &&& \tyarr{\tau_1}{\tau_2}  & \tau_1 \rightarrow \tau_2 & \text{function}
  \\
  \\ 
  \EXPS & e & \Coloneqq & \var{x} & \var{x} & \text{variable}
  \\
         &&& \ez &  \ez & \text{zero}
  \\
         &&& \esucc{e} &  \esucc{e} & \text{successor}
  \\
         &&& \elam{\tau}{x}{e} & \clam{\tau}{x}{e} & \text{abstraction}
  \\
         &&& \eapp{e_1}{e_2} & \capp{e_1}{e_2} & \text{application}
  \\
         &&& \eiter{e_0}{x}{e_1}{e} & {\citer{e_0}{x}{e_1}{e}}   & \text{iterator}
  \\
         &&& \HL{[\,]} & \HL{[\,]} & \text{empty list}
  \\
         &&& \HL{\econs{e}{e'}} & \HL{\ccons{e}{e'}} & \text{list constructor}
  \\                                   
         &&& \HL{\edeclist{e_0}{x}{y}{e_1}{e}} & \HL{{\cdeclist{e_0}{x}{y}{e_1}{e}}}   &\text{list deconstructor}
\end{array}
\]

The construct $\econs{e}{e'}$ simply builds a new list from an element
$e$ and another list $e'$ by adding $e$ at the front of $e'$.
%
The construct $\edeclist{e_0}{x}{y}{e_1}{e}$ is a
\emph{binder} which introduces two new variables ($\var{x}$ and
$\var{y}$).
%
This construct should behave differently when $e$ is the empty list
(behave as $e_0$). If $e$ is non-empty, then it is divided into a head
(bound to $\var{x}$) and a tail (bound to $\var{y}$).

\section{Exercise 1: add typing rules for the new syntax constructs}
Here are the typing rules for $\lggeT$, add new rules to support lists.
\[
\tyrule
{\ttyrulename{var}}
{\,}
{\tyjudge{\Gamma_1, \var{x}: \tau, \Gamma_2}{\var{x}}{\tau} }
\qquad\qquad\qquad
\tyrule
{\ttyrulename{nat}}
{\,}
{\tyjudge{\Gamma}{\ez}{\tynat}}
\qquad\qquad\qquad
\tyrule
{\ttyrulename{num}}
{\tyjudge{\Gamma}{{e}}{\tynat{}}}
{\tyjudge{\Gamma}{\esucc{e}}{\tynat{}}}
\]
% 
\[
\tyrule
{\ttyrulename{lam}}
{\tyjudge{\Gamma, \var{x} : \tau_1}{e}{\tau_2}}
{\tyjudge{\Gamma}{\elam{\tau_1}{x}{e}}{\tyarr{\tau_1}{\tau_2}}}
\qquad\qquad\qquad
\tyrule
{\ttyrulename{ap}}
{
  \tyjudge{\Gamma}{e_1}{\tyarr{\tau_1}{\tau}}
  \and
  \tyjudge{\Gamma}{e_2}{\tau_1}
}
{
  \tyjudge{\Gamma}{\eapp{e_1}{e_2}}{\tau}
}
\]
\[
\tyrule
{\ttyrulename{ite}}
{
  \tyjudge{\Gamma}{e}{\tynat}
  \and
  \tyjudge{\Gamma}{e_0}{\tau}
  \and
  \tyjudge{\Gamma, \var{x} : \tau}{e_1}{\tau}
}
{
  \tyjudge{\Gamma}{\eiter{e_0}{x}{e_1}{e}}{\tau}
}
\]


\section{Exercise 2: add semantic rules for the new syntax constructs}


Here are the evaluation rules for $\lggeT$, add new rules to support lists.

% \subsection{Values}
\[
\semrule{\tsemrulename{z}}
{\,}
{\valjudge{\ez}}
\qquad \qquad  \qquad
\semrule{\tsemrulename{s}}
{\valjudge{e}}
{\valjudge{\esucc{e}}}
\qquad \qquad  \qquad
\semrule
{\tsemrulename{lam}}
{\,}
{\valjudge{\elam{\tau}{x}{e}}}
\]



% \subsection{Reductions}

\[
\semrule
{\tsemrulename{ss}}
{\jtrans{e}{e'}}
{
  \jtrans{\esucc{e}}{\esucc{e'}}
}
\qquad \qquad  \qquad
% 
\semrule
{\tsemrulename{ap}}
{\jtrans{e_1}{e'_1}}
{
  \jtrans{\eapp{e_1}{e_2}}{\eapp{e'_1}{e_2}}
}
\]
% 
% 
\[
\semrule
{\tsemrulename{lan}}
{\valjudge{e_1} 
  \and \jtrans{e_2}{e'_2}
}
{\jtrans{\eapp{e_1}{e_2}}{\eapp{e_1}{e'_2}}}
\qquad \qquad  \qquad
\semrule
{\tsemrulename{lav}}
{\valjudge{e_2}}
{\jtrans{\eapp{\elam{\tau}{x}{e_1}}{e_2}}{\subs{e_1}{e_2}{x}}}
\]
% 
\[
\semrule
{\tsemrulename{rin}}
{\jtrans{e}{e'}}
{
  \jtrans{\eiter{e_0}{x}{e_1}{e}}{\eiter{e_0}{x}{e_1}{e'}}
}
\qquad \qquad  \qquad
% 
\semrule
{\tsemrulename{r0}}
{\,}
{
  \jtrans{\eiter{e_0}{x}{e_1}{\ez}}{e_0}
}
% 
\]
\[
%  
\semrule
{\tsemrulename{rs}} 
{\valjudge{{e}}}
{
  \jtrans{\eiter{e_0}{x}{e_1}{\esucc{e}}}{\subs{e_1}{{\eiter{e_0}{x}{e_1}{{e}}}}{x}}
}
\]


\section*{\LaTeX\ template}
The \LaTeX\ sources of this document are available in:
\url{https://github.com/ukc-co663/pl-design-latex-template}.

% \end{document}

\section{Model answer}

\subsection{Typing rules}
\[
\tyrule
{\ttyrulename{dlist}}
{
  \tyjudge{\Gamma}{e}{\tylist{\tau}}
  \and
  \tyjudge{\Gamma}{e_0}{\tau'}
  \and
  \tyjudge{\Gamma, \var{x} : \tau, \var{y}: {\tylist{\tau}}}{e_1}{\tau'}
}
{
  \tyjudge{\Gamma}{\edeclist{e_0}{x}{y}{e_1}{e}}{\tau'}
}
\]



\[
\tyrule
{\ttyrulename{e}}
{\,}
{
  \tyjudge{\Gamma}{[\,]}{\tylist{\tau}}
}
\qquad \qquad  \qquad \qquad
\tyrule
{\ttyrulename{cons}}
{
  \tyjudge{\Gamma}{e}{{\tau}}
  \and
  \tyjudge{\Gamma}{e'}{\tylist{\tau}}
}
{
  \tyjudge{\Gamma}{\econs{e}{e'}}{\tylist{\tau}}
}
\]


\subsection{Semantic rules}

\[
\semrule{\tsemrulename{cons}}
{\valjudge{e}
  \and 
  \valjudge{e'}
}
{\valjudge{\econs{e}{e'}}}
\qquad \qquad  \qquad \qquad
\semrule{\tsemrulename{empty}}
{\,
}
{\valjudge{[\,]}}
\]

\[
\semrule
{\tsemrulename{cin}}
{\jtrans{e}{e'}}
{
  \jtrans{\edeclist{e_0}{x}{y}{e_1}{e}}{\edeclist{e_0}{x}{y}{e_1}{e'}}
}
\qquad \qquad  \qquad
% 
\semrule
{\tsemrulename{c0}}
{\,}
{ 
  \jtrans{\edeclist{e_0}{x}{y}{e_1}{[\,]}}{e_0}
}
% 
\]
\[
%  
\semrule
{\tsemrulename{cs}}
{\,}
{
  \jtrans{\edeclist{e_0}{x}{y}{e_1}{\econs{e}{e'}}}
  {\twosubs
    {e_1}
    {e}{x}
    {e'}{y}
  }
}
\]


\end{document}


%%% Local Variables:
%%% mode: latex
%%% TeX-master: t
%%% End:
