\documentclass[11pt]{article}
\usepackage[top=2cm,bottom=2cm]{geometry}
\usepackage[T1]{fontenc}
\usepackage{url}
\usepackage{hyperref}
\usepackage{xcolor}
% 
\usepackage{amsthm}
\usepackage{amsmath}
\usepackage{amssymb}
\usepackage{ifthen}
\usepackage{mathpartir}
\usepackage{mathtools}
\input{../../slides/macros}
% 
\begin{document}

\title{CO663 Part II: Worksheet 3 (\emph{assessed})}

\date{\vspace{-5ex}}
\maketitle

% \pagenumbering{gobble}

\newcommand{\answerbox}[1]{\framebox{\parbox[c][#1]{\textwidth}{
      \color{white}{h}% 
    }}}


\section{Introduction}

The objective of this class is to put in practice the typing and
semantic rules for language $\lggeE$.
% 
This is an individual exercise.

\begin{center}
  \emph{This class assessment is worth {\color{red} 6} marks.}
\end{center}

Submit your work via Moodle (within the deadline specified on the
Moodle page).


\section{Exercises}

\begin{enumerate}
\item Formally type-check \emph{and} evaluate (until your reach a value) the
following expressions:
\begin{enumerate}
\item $\elet{\enum{3}}{{z}}{\etimes{\var{z}}{\enum{2}}}$ (its type is $\enum{}$)
\item $\eplus{\etimes{\enum{12}}{\enum{0}}}{\elen{\estr{hi}}}$ (its type is $\enum{}$).
\end{enumerate}
\item Write down your own syntactically correct and well-typed term
  (it must be have a \emph{depth of 3}, at least). Show
  that is well-typed and evaluate it fully.
\end{enumerate} 

Each (type or semantic) derivation is worth 1 mark.

% 

\section{Additional information}
The depth of term $e$, written $\lvert e \rvert$, is the depth of its
abstract syntax tree.
% 
It is is defined formally as follows:
% 
\[
\lvert e \rvert =
\begin{cases}
  1 & \text{if } e = \var{x}, \ e = \enum{n}, \ e = \estr{s}, \ e = \ktrue, \ \text{or } e = \kfalse
  \\
  1 {+}  \lvert e' \rvert  & \text{if } e = \elen{e'}
  \\
  1   {+} \textit{max}(\lvert e_1 \rvert, \lvert e_2 \rvert)  & \text{if }  e = \elet{e_1}{x}{e_2}, \ e = \etimes{e_1}{e_2}, \  e = \eplus{e_1}{e_2}, \ \text{etc.}
  \\
  1   {+} \textit{max}(\lvert e' \rvert, \lvert e_1 \rvert, \lvert e_2 \rvert)  & \text{if } e = \site{e'}{e_1}{e_2}
\end{cases}
\]
For instance,
\begin{itemize}
\item $\lvert \elet{\enum{3}}{{z}}{\etimes{\var{z}}{2}} \rvert = 3$, 
\item $\lvert \eplus{\etimes{\enum{12}}{\enum{0}}}{\elen{\estr{hi}}} \rvert = 3$.
\end{itemize}

\section*{\LaTeX\ template}

If you would like to use \LaTeX\ to typeset your solution, you can use
the template here:
\url{https://github.com/ukc-co663/pl-design-latex-template}.


\end{document}

\appendix

\section{Model answer}

\subsection{Exercise 2.1.a}

\paragraph{Typing}

\[
\tyrule{let}
{
  \tyrule{num}
  {\,}
  {
    \tyjudge{\varnothing}{\enum{3}}{\enum{}}
  }
  \quad
  \\
  \tyrule{times}
  {
    \tyrule{var}
    {\,}
    {
      \tyjudge{\var{z} : \enum{}}{\var{z}}{\enum{}}
    }
    \\
    \tyrule{num}
    {\,}
    {
      \tyjudge{\var{z} : \enum{}}{\enum{2}}{\enum{}}
    }
  }
  {
    \tyjudge
    {\var{z} : \enum{}}
    {\etimes{\var{z}}{\enum{2}}}
    {\enum{}}
  }
}
{
  \tyjudge{\varnothing}
  {\elet{\enum{3}}{{z}}{\etimes{\var{z}}{\enum{2}}}}
  {\enum{}}
}
\]

\paragraph{Evaluation}

\begin{enumerate}
\item First step/tree:
  \[
  \semrule
  {letn}
  {\,}
  {\jtrans
    {\elet{\enum{3}}{{z}}{\etimes{\var{z}}{\enum{2}}}}
    {\etimes{\enum{3}}{\enum{2}}}
  }
  \]
\item Second step/tree:
  \[
  \semrule
  {letn}
  {3 \times 2 = 6 }
  {\jtrans
    {\etimes{\enum{3}}{\enum{2}}}
    {\enum{6}}
  }
  \]
\end{enumerate}

\subsection{Exercise 2.1.b}

\paragraph{Typing}

\[
\tyrule{plus}
{
  \tyrule{times}
  {
    \tyrule{num}
    {\,}
    {  \tyjudge{\varnothing}
      {\enum{12}}
      {\enum{}}}
    \\
    \tyrule{num}
    {\,}
    {  \tyjudge{\varnothing}
      {\enum{0}}
      {\enum{}}}
  }
  {
    \tyjudge{\varnothing}
    {\etimes{\enum{12}}{\enum{0}}}
    {\enum{}}
  }
  \\
  \tyrule{len}
  {
    \tyrule{str}
    {\,}
    {
      \tyjudge{\varnothing}
      {\estr{hi}}
      {\estr{}}
    }
  }
  {
    \tyjudge{\varnothing}
    {\elen{\estr{hi}}}
    {\enum{}}
  }
}
{
  \tyjudge{\varnothing}
  {\eplus{\etimes{\enum{12}}{\enum{0}}}{\elen{\estr{hi}}}}
  {\enum{}}
}
\]
\paragraph{Evaluation}


\begin{enumerate}
\item First step/tree:
  \[
  \semrule
  {pll}
  {
    \semrule
    {tmval}
    {12 \times 0 = 0}
    {
      \jtrans
      {\etimes{\enum{12}}{\enum{0}}}
      {\enum{0}}
    }
  }
  {\jtrans
    {\eplus{\etimes{\enum{12}}{\enum{0}}}{\elen{\estr{hi}}}}
    {\eplus{\enum{0}}{\elen{\estr{hi}}}}
  }
  \]
\item Second step/tree:
  \[
  \semrule
  {plr}
  {
    % 
    {\valjudge{\enum{0}}}
    \\\qquad
    \semrule
    {lenv}
    {\lvert hi \rvert = 2}
    {
      \jtrans
      {\elen{\estr{hi}}}
      {\enum{2}}
    }
  }
  {\jtrans
    {\eplus{\enum{0}}{\elen{\estr{hi}}}}
    {\eplus{\enum{0}}{\enum{2}}}
  }
  \]
\item Third step/tree:
  \[
  \semrule
  {plval}
  {0+2=2}
  {\jtrans
    {\eplus{\enum{0}}{\enum{2}}}
    {\enum{2}}
  }
  \]
  
\end{enumerate}



\end{document}


%%% Local Variables:
%%% mode: latex
%%% TeX-master: t
%%% End:
