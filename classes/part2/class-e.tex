\documentclass[11pt]{article}
\usepackage[top=2cm,bottom=2cm]{geometry}
\usepackage[T1]{fontenc}
\usepackage{url}
\usepackage{hyperref}
\usepackage{xcolor}
% 
\usepackage{amsthm}
\usepackage{amsmath}
\usepackage{amssymb}
\usepackage{ifthen}
\usepackage{mathpartir}
\usepackage{mathtools}
\input{../../slides/macros}
% 
\begin{document}

\title{CO663 Part II: Worksheet 1}

\date{\vspace{-5ex}}
\maketitle

% \pagenumbering{gobble}

\newcommand{\answerbox}[1]{\framebox{\parbox[c][#1]{\textwidth}{
      \color{white}{h}% 
    }}}



\section{Introduction}

The objective of this class is to put in practice the typing rules for
language $\lggeE$.
% 

\section{Language $\lggeE$}

Here is the syntax and type system for language $\lggeE$ (same as in
the lecture slides).

\subsection{Syntax}

  \[
  \begin{array}{llclll}
    \TYPES & \tau & \Coloneqq & \enum{}  & \enum{} & \text{numbers}
    \\
           &&& \estr{} & \estr{} & \text{strings}
    \\
    \\ 
    % 
    \EXPS & e & \Coloneqq  & \var{x} & \var{x} & \text{variable}
    \\
           &&& \enum{n} & n & \text{numeral}
    \\
           &&& \estr{s} & \litstr{s} & \text{literal}
    \\
           &&& \eplus{e_1}{e_2} & e_1 {+} e_2 & \text{addition}
    \\
           &&& \etimes{e_1}{e_2} & e_1 {*} e_2 & \text{multiplication}
    \\
           &&& \ecat{e_1}{e_2} & e_1 \concat e_2 & \text{concatenate}
    \\
           &&& \elen{e} & \lvert e \rvert & \text{length}
    \\
           &&& \elet{e_1}{x}{e_2} & \clet{e_1}{x}{e_2} & \text{definition}
  \end{array}
  \]
\subsection{Typing rules}

  \[
  \tyrule
  {\etyrulename{var}}
  {\,}
  {\tyjudge{\Gamma_1, \var{x}: \tau, \Gamma_2}{\var{x}}{\tau} }
  \]
  \[
  \tyrule
  {\etyrulename{str}}
  {\,}
  {\tyjudge{\Gamma}{\estr{s}}{\estr{}}}
  \qquad\qquad\qquad
  \tyrule
  {\etyrulename{num}}
  {\,}
  {\tyjudge{\Gamma}{\enum{n}}{\enum{}}}
  % \end{array}
  \]
  \[
  \tyrule
  {\etyrulename{plus}}
  {
    \tyjudge{\Gamma}{e_1}{\enum{}}
    \and
    \tyjudge{\Gamma}{e_2}{\enum{}}
  }
  {\tyjudge{\Gamma}{\eplus{e_1}{e_2}}{\enum{}}}
  \qquad\qquad\qquad
  \tyrule
  {\etyrulename{times}}
  {
    \tyjudge{\Gamma}{e_1}{\enum{}}
    \and
    \tyjudge{\Gamma}{e_2}{\enum{}}
  }
  {\tyjudge{\Gamma}{\etimes{e_1}{e_2}}{\enum{}}}
  % 
  \]
  \[
  \tyrule
  {\etyrulename{cat}}
  {
    \tyjudge{\Gamma}{e_1}{\estr{}}
    \and
    \tyjudge{\Gamma}{e_2}{\estr{}}
  }
  {\tyjudge{\Gamma}{\ecat{e_1}{e_2}}{\estr{}}}
  \qquad\qquad\qquad
  \tyrule
  {\etyrulename{len}}
  {
    \tyjudge{\Gamma}{e}{\estr{}}
  }
  {\tyjudge{\Gamma}{\elen{e}}{\enum{}}}
  \]
  \[
  \tyrule
  {\etyrulename{let}}
  {
    \tyjudge{\Gamma}{e_1}{\tau_1}
    \and
    \tyjudge{\Gamma, \var{x} : \tau_1 }{e_2}{\tau}
  }
  {\tyjudge{\Gamma}{\elet{e_1}{x}{e_2}}{\tau}}
  \]

\section{Exercises}

\subsection{Typing derivation}

Perform the full typing derivation for the following $\lggeE$ programs:
  \begin{itemize}
  \item $\ecat
    {\ecat
      {\estr{\text{hello}}}   
      {\estr{\text{world}}}
    }
    {\estr{\text{!}}}$ --- start with type $\estr{}$.
    % 
  \item $\etimes
    {\elen{\estr{\text{hello}}}}
    {\eplus{\enum{1}}{\elen{\estr{\text{world}}}}}$
    --- start with type $\enum{}$.
    % 
  \item $\elet{\estr{\text{mystring}}}{x}{\etimes{\elen{\var{x}}}{\enum{0}}}$
    --- start with type $\enum{}$.
    % 
  \item $\elet
    {\estr{\text{anotherstring}}}
    {y}
    {\eplus{\var{y}}{\etimes{\enum{1}}{\enum{2}}}}$
    --- start with type $\enum{}$.
    % 
  \item $\elet
    {\ecat{\estr{\text{anotherstring}}}{\var{y}}}
    {y}
    {\var{y}}$
    --- start with type $\estr{}$.
  \end{itemize}


  For instance, to check that program
  $\elet{\enum{42}}{z}{\eplus{\var{z}}{1}}$ has type $\enum{}$, you
  should construct the following tree (starting from the bottom,
  applying the appropriate rule at each node of the tree):
  
  {\small
  \[
  \tyrule{\etyrulename{let}}
  {
    \tyrule{\etyrulename{num}}
    {\,}
    {
      \tyjudge{\emptyset}
      {\enum{42}}
      {\enum{}}
    }
    \\
    \qquad   \qquad
    \tyrule{\etyrulename{plus}}
    {
      \tyrule{\etyrulename{var}}
      {\,}
      {
        \tyjudge{\var{z} : \enum{}}
        {\var{z}}
        {\enum{}}
      }
      \\
      \qquad   \qquad
      \tyrule{\etyrulename{num}}
      {\,}
      {
        \tyjudge{\var{z} : \enum{}}
        {\enum{1}}
        {\enum{}}
      }
    }
    { 
      \tyjudge{\var{z} : \enum{}}
      {\eplus{\var{z}}{\enum{1}}}
      {\enum{}}
    }
  }
  {
    \tyjudge{\emptyset}
    {\elet{\enum{42}}{z}{\eplus{\var{z}}{\enum{1}}}}
    {\enum{}}
  }
  \]
  }

\subsection{Language extension}

Currently $\lggeE$ does not feature any conditional construct. What
would you need to add to the language (in terms of syntax and typing
rules) so that it supports a construct of the form:
\[
\csite{e}{e_1}{e_2}
\]


\section*{\LaTeX\ template}

If you would like to use \LaTeX\ to typeset your solution, you can use
the template here:
\url{https://github.com/ukc-co663/pl-design-latex-template}.

\end{document}


%%% Local Variables:
%%% mode: latex
%%% TeX-master: t
%%% End:
