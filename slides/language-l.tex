\section{Resource-awareness}

\subsection{Syntax}

\begin{frame}[fragile]
  \frametitle{Introduction}

  Resource-awareness in Rust, via ownership types.
\begin{lstlisting}
fn main() {
    let i = Box::new(513i32);
    foo(i);
    foo(i);
}

fn foo(i: Box<i32>) {
    println!("i is: {}", i);
}
\end{lstlisting}

\end{frame}


\begin{frame}[fragile]
\frametitle{Introduction}
The compiler complains with
\begin{lstlisting}
error[E0382]: use of moved value: `i`
 --> hello.rs:4:9
  |
3 |     foo(i);
  |         - value moved here
4 |     foo(i);
  |         ^ value used here after move
  |
 = note: move occurs because `i` has type `std::boxed::Box<i32>`,
         which does not implement the `Copy` trait

error: aborting due to previous error
\end{lstlisting}

\pause

  \begin{quotation}
    Ownership rules ensure, that at any point, for a single
    non-copyable value, there is \emph{only one owner} that can change
    it.
  \end{quotation}

\end{frame}


\note{The main goal of ownership types in Rust is to have memory
  safety without garbage collection}

\begin{frame}
  \frametitle{Language $\lggeL$}
  \[
  \begin{array}{l@{\;}l@{\;}c@{\;}l@{\;}l@{\;}l}
    \QUAL & q & \Coloneqq & \unq & \unq & \text{unrestricted}
    \\ 
          &&& \linq & \linq & \text{linear}
    \\\\
    \PTYPES & \rho & \Coloneqq & \ebool{} & \ebool{} & \text{boolean}
    \\
          &&& \typrod{\tau_1}{\tau_2} & \tau_1 \times \tau_2 & \text{product} 
    \\
           &&& \tyarr{\tau_1}{\tau_2}  & \tau_1 \rightarrow \tau_2 & \text{function}
    \\\\
    \TYPES & \tau & \Coloneqq & \ptype{q}{\rho}  & \ptype{q}{\rho}  & \text{qualified pre-type}
    \\\\
    \EXPS & e & \Coloneqq & \var{x} & \var{x} & \text{variable}
    \\
           &&& \elbool{b}{q} & \ptype{q}{b} & \text{boolean}
    \\
          &&& \elpair{e_1}{e_2}{q} & \clpair{e_1}{e_2}{q} & \text{pair}
    \\
           &&& \ellam{\tau}{x}{e}{q} & \cllam{\tau}{x}{e}{q} & \text{abstraction}
    \\
           &&& \eapp{e_1}{e_2} & \capp{e_1}{e_2} & \text{application}
    \\
          &&& \esplit{e_1}{x}{y}{e_2} & \csplit{e_1}{x}{y}{e_2} & \text{split}
  \end{array}
  \]
\end{frame}


\begin{frame}
  \frametitle{Operations on contexts}

  We revise our notion of context in this part so that $\Gamma$ is of the form:
  \[
  \begin{array}{lcll}
    \Gamma & \Coloneqq & \varnothing & \text{empty context}
    \\
           && \Gamma, \var{x}: \ptype{q}{\tau} & \text{hypothesis}
  \end{array}
  \]


  \begin{itemize}
  \item The context $\varnothing$ does not contain any hypothesis.
  \item Example of $\Gamma$:
    \[\var{x} : \ptype{\linq}{\ebool{}}, \var{y} :
    \ptype{\unq}{\ebool{}}, \var{z}:
    \ptype{\unq}{\tyarr{\ebool{}}{\ebool{}}}
    \]
  \end{itemize}
\end{frame}


\begin{frame}
  \frametitle{Splitting contexts}
  We define rules to split the context according to the restriction
  of the hypotheses therein.

  \[
  \tyrule{empty}
  {\,}
  {\varnothing = \varnothing \ctxtsplit \varnothing}
  \]
  
  \[
  \tyrule{unres}
  {\Gamma = \Gamma_1 \ctxtsplit \Gamma_2}
  {\Gamma, \var{x}: \ptype{\unq}{\rho} =
    (\Gamma_1, \var{x}: \ptype{\unq}{\rho}) \ctxtsplit (\Gamma_2, \var{x}: \ptype{\unq}{\rho}) }
  \]

  \[
  \tyrule{res-l}
  {\Gamma = \Gamma_1 \ctxtsplit \Gamma_2}
  {\Gamma, \var{x}: \ptype{\linq}{\rho} =
    (\Gamma_1, \var{x}: \ptype{\linq}{\rho}) \ctxtsplit \Gamma_2 }
  \]

  \[
  \tyrule{res-r}
  {\Gamma = \Gamma_1 \ctxtsplit \Gamma_2}
  {\Gamma, \var{x}: \ptype{\linq}{\rho} =
    \Gamma_1 \ctxtsplit
    (\Gamma_2, \var{x}: \ptype{\linq}{\rho}) }
  \]

\end{frame}


\subsection{Type system}

\begin{frame}
  \frametitle{Statics}
  \[
  \tyrule
  {\ltyrulename{var}}
  {
    \unres{\Gamma_1} 
    \and
    \unres{\Gamma_2}
  }
  {\tyjudge{\Gamma_1, \var{x}: \tau , \Gamma_2}{\var{x}}{\tau}}
  \]

  \[
  \tyrule
  {\ltyrulename{bool}}
  {\unres{\Gamma}}
  {\tyjudge{\Gamma}{\elbool{b}{q}}{\elbool{}{q}}}
  \]

  \bigskip

  Auxiliary definitions:
  \begin{itemize}
  \item $\unres{\tau}$ if and only if $\tau = \ptype{\unq}{\rho}$
  \item $\linear{\tau}$ always holds (i.e., $\tau =  \ptype{\unq}{\rho}$ or  $\tau =  \ptype{\linq}{\rho}$)
  \item $q(\Gamma)$ if and only if $(\var{x} : \tau) \in \Gamma$ implies $q(\tau)$. 
    % 
  \end{itemize}
\end{frame}




\begin{frame}
  \frametitle{Statics}
  

  \[
  \tyrule
  {\ltyrulename{pair}}
  {
    q(\tau_1)
    \and
    \tyjudge{\Gamma_1}{e_1}{\tau_1}
    \and
    q(\tau_2)
    \and
    \tyjudge{\Gamma_2}{e_2}{\tau_2}
  }
  {
    \tyjudge{\Gamma_1 \ctxtsplit \Gamma_2}{\elpair{e_1}{e_2}{q}}{\ptype{q}{\typrod{\tau_1}{\tau_2}}
    }
  }
  \]

  \bigskip

  \[
  \tyrule
  {\ltyrulename{split}}
  {
    \tyjudge{\Gamma_1}{e_1}{\ptype{q}{\typrod{\tau_1}{\tau_2}}}
    \and
    \tyjudge{\Gamma_2, \var{x}: \tau_1, \var{y}: \tau_2 }{e_2}{\tau}
    }
    {
    \tyjudge{\Gamma_1 \ctxtsplit \Gamma_2}
    {\esplit{e_1}{x}{y}{e_2}}
    {\tau}
  }
  \]
  

\end{frame}




\begin{frame}
  \frametitle{Statics}
  

  \[
  \tyrule
  {\ltyrulename{abs}}
  {
    q(\Gamma)
    \and
    \tyjudge{\Gamma, \var{x} : \tau_1}{e}{\tau_2}
  }
  {
    \tyjudge{\Gamma}
    {\ellam{\tau_1}{x}{e}{q}}
    {\ptype{q}{\tyarr{\tau_1}{\tau_2}}}
  }
  \]

  \bigskip

  \[
  \tyrule
  {\ltyrulename{ap}}
  {
    \tyjudge{\Gamma_1}{e_1}{\ptype{q}{\tyarr{\tau_1}{\tau_2}}}
    \and
    \tyjudge{\Gamma_2}{e_2}{\tau_2}
    }
    {
    \tyjudge{\Gamma_1 \ctxtsplit \Gamma_2}
    {\eapp{e_1}{e_2}}
    {\tau_2}
  }
  \]
 
\end{frame}


%  $\cllam{\tau}{x}{e}{q}$
%  \capp{e_1}{e_2} 
%  \clpair{e_1}{e_2}{q}
\begin{frame}
  \frametitle{Examples}
  %  Explain different restrictions with examples (see Walker) 
  Note the
  premise $q(\Gamma)$ in rule $\grulename{\ltyrulename{abs}}$.
  % 
  % 
  Consider the following examples:


  \[
  \begin{array}{rcl}
    T & \defi &\ptype{\unq}{\ebool{}} \rightarrow \ptype{\linq}{\ebool{}}
    \\
    A & \defi &\cllam{\ptype{\linq}{\ebool{}}}{x}{
                \capp
                {\cllam{\ptype{\unq}{T}}{f}{\ptype{\linq}{\ktrue}}{\linq}}
                {\cllam{\ptype{\unq}{\ebool{}}}{y}{\var{x}}{\unq}}
                }
                {\linq}
    \\
    B & \defi & \cllam{\tau}{x}{
    \capp
    {\cllam{\ptype{\unq}{T}}{f}{ 
                \clpair
                {E}
                {E}
                {\linq}
                }{\linq}}
    {\cllam{\ptype{\unq}{\ebool{}}}{y}{\var{x}}{\unq}}
    }{\linq}
    \\
    && \text{with} \quad E \defi {\capp{\var{f}}{\ptype{\unq}{\ktrue}}}
  \end{array}
  \]
  % 
\end{frame}




\begin{frame}
  \frametitle{Type safety}
  What sort of additional type safety guarantees does this language
  give us?

  \bigskip
  How to formalise them?
\end{frame}


\begin{frame}
  \frametitle{Linearity in PL: references}
  \begin{itemize}
  \item
    \href{https://mitpress.mit.edu/sites/default/files/titles/content/9780262162289_sch_0001.pdf}{Substructural
      Type System}, by David Walker (source material)
%
  \item
    \href{https://www.cs.cmu.edu/~./fp/courses/15816-f01/handouts/linlam.pdf}{Linear
      $\lambda$-Calculus}, by Frank Pfenning
  \item
    \href{http://nercury.github.io/rust/guide/2015/01/19/ownership.html}{Quick
      overview of ownership types in Rust}
  \end{itemize}
\end{frame}


%%% Local Variables:
%%% mode: latex
%%% TeX-master: "main"
%%% End:
