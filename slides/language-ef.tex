\section{Higher-order functions}



\begin{frame}{Remember $\lggeE$?}
    \[
  \begin{array}{llclll}
    \TYPES & \tau & \Coloneqq & \enum{}  & \enum{} & \text{numbers}
    \\
           &&& \estr{} & \estr{} & \text{strings}
    \\
    \\  \pause
    % 
    \EXPS & e & \Coloneqq  & \var{x} & \var{x} & \text{variable}
    \\
           &&& \enum{n} & n & \text{numeral}
    \\
           &&& \estr{s} & \litstr{s} & \text{literal}
    \\
           &&& \eplus{e_1}{e_2} & e_1 {+} e_2 & \text{addition}
    \\
           &&& \etimes{e_1}{e_2} & e_1 {*} e_2 & \text{multiplication}
    \\
           &&& \ecat{e_1}{e_2} & e_1 \concat e_2 & \text{concatenate}
    \\
           &&& \elen{e} & \lvert e \rvert & \text{length}
    \\
           &&& \elet{e_1}{x}{e_2} & \clet{e_1}{x}{e_2} & \text{definition}
  \end{array}
  \]



%%% Local Variables:
%%% mode: latex
%%% TeX-master: "main"
%%% End:
 
\end{frame}


\begin{frame}
  \frametitle{Introduction}
  In $\lggeE$, to multiply a number by $2$, we can write:
  \[
  \clet{4}{x}{\etimes{\enum{2}}{\enum{x}}}
  \]
  whenever we want to double a number with have to ``redefine''
  the concept of doubling.


  \bigskip

  What if I don't want to write the concept of ``doubling'' several times in, e.g.,
  $(10\times2)+(4\times2)+(5\times2)$?
  
  \bigskip

  \pause

  \bigskip

  What concept(s) are we missing?
\end{frame}



\subsection{Syntax and types}

\begin{frame}[fragile]
  \frametitle{Language $\lggeEF$}

  \pause

  \[
  \begin{array}{llclll}
    \TYPES & \tau & \Coloneqq & \ldots {\color{gray} (\text{as in } \lggeE)}
    \\ 
           &&& \tyarr{\tau_1}{\tau_2}  & \tau_1 \rightarrow \tau_2 & \text{function}
    \\ 
    \\\pause
    \EXPS & e & \Coloneqq & \ldots {\color{gray} (\text{as in } \lggeE)}
    \\
           &&& \elam{\tau}{x}{e} & \clam{\tau}{x}{e} & \text{abstraction}
    \\
           &&& \eapp{e_1}{e_2} & \capp{e_1}{e_2} & \text{application}
  \end{array}
  \]

  \bigskip
  \pause

  In other programming languages:
\begin{lstlisting}
  (Object x) -> x.getClass() // Java

  \x -> x + 1                -- Haskell
\end{lstlisting}

  \pause

  \[
  \quad
  \begin{array}{ll}
    \text{Example:} \quad &
                            \cleta{\clam{\enum{}}{x}{\etimes{\var{x}}{\enum{2}}}}{double}
    \\
                          & 
                            \quad
                            \cletb
                            { 
                            \eplus{
                            \capp{\var{double}}{\enum{4}} 
                            }
                            {
                            \capp{\var{double}}{{\enum{10}}} 
                            }
                            }
  \end{array}
  \]
\end{frame}




\begin{frame}
  \frametitle{Statics}
  \[
  \tyrule
  {\eftyrulename{lam}}
  {
    \only<2->{
      \tyjudge{\Gamma, \var{x} : \tau_1}{e}{\tau_2}
    }
  }
  {\tyjudge{\Gamma}{\elam{\tau_1}{x}{e}}{\tyarr{\tau_1}{\tau_2}}}
  \]

  
  \bigskip


  \pause

  \pause
  
  \[
  \tyrule
  {\eftyrulename{ap}}
  {
    \only<4->{
      \tyjudge{\Gamma}{e_1}{\tyarr{\tau'}{\tau}}
    \and
      \tyjudge{\Gamma}{e_2}{\tau'}
    }
  }
  {
    \tyjudge{\Gamma}{\eapp{e_1}{e_2}}{\tau}
  }
  \]

\end{frame}


\subsection{Semantics}



\begin{frame}
  \frametitle{Examples}
  \label{fr:ef-types}
  Are these $\lggeEF$ programs well typed?


  {\footnotesize
    \begin{itemize}[<+->]
    \item $\elet{\clam{\enum{}}{x}{\etimes{\var{x}}{\var{x}}}}{sq}
      { 
        \eplus{
          \eapp{\var{sq}}{4} 
        }
        {
          \eapp{\var{sq}}{10} 
        }
      }$
      % 
    \item { $\elet{\clam{\enum{}}{x}{
            \clam{\enum{}}{y}{
              \etimes{\var x}{\var y}
            }
          }
        }{f}{
          \eapp
          {\eapp
            {\var{f}}
            {\enum{4}}
          }
          {\enum{4}}
        }
        $}
      
    \item $\elet{\clam{\enum{}}{x}{
          \eapp{\var f}{\eplus{\var{x}}{\enum{1}}}
        }
      }{f}{
        \eapp{\var{f}}{0}
      }$ 
    \end{itemize}
  }

\end{frame}



\begin{frame}
  \frametitle{Dynamics}

  First: define what a value is
  \pause

  \bigskip 
  \[
  \semrule
  {\efsemrulename{lam}}
  {\,}
  {\valjudge{\elam{\tau}{x}{e}}}
  \]
  
  \bigskip 
  \pause
  Second: define evaluation judgements
  \pause

  \[
  \semrule
  {\efsemrulename{ap}}
  {\jtrans{e_1}{e'_1}}
  {
    \jtrans{\eapp{e_1}{e_2}}{\eapp{e'_1}{e_2}}
  }
  \]

  \bigskip

  \pause

  \bigskip  

  We write $\jtransC{e_1}{e_n}$ if $\jtrans{e_1}{e_2}$,
  $\jtrans{e_2}{e_3}$, \ldots, $\jtrans{e_{n-1}}{e_n}$ for some
  terms $e_1$, \ldots, $e_{n-1}$.

\end{frame}

\begin{frame}
  \frametitle{Dynamics}

  \begin{itemize}[<+->]
  \item Call-by-name
    \[
    \semrule
    {\efsemrulename{apn}}
    {\,}
    {\jtrans{\eapp{\elam{\tau}{x}{e_1}}{e_2}}{\subs{e_1}{e_2}{x}}}
    \]
    % 
  \item Call-by-value

    \[
    \semrule
    {\efsemrulename{apr}}
    {\valjudge{e_1} 
      \and \jtrans{e_2}{e'_2}
    }
    {\jtrans{\eapp{e_1}{e_2}}{\eapp{e_1}{e'_2}}}
    \]

    
    \[
    \semrule
    {\efsemrulename{apl}}
    {\valjudge{e_2}}
    {\jtrans{\eapp{\elam{\tau}{x}{e_1}}{e_2}}{\subs{e_1}{e_2}{x}}}
    \]
  \end{itemize}

  \bigskip
  % 

  What are the differences between call-by-name and call-by-value?
\end{frame}

\begin{frame}
  \frametitle{Examples}
  \label{fr:evaluate-ef}
  Reduce the following expression (if they are typeable!):

  {\footnotesize
    \begin{itemize}[<+->]
    \item $\elet{\clam{\enum{}}{x}{\etimes{\var{x}}{\var{x}}}}{sq}
      { 
        \eplus{
          \eapp{\var{sq}}{4} 
        }
        {
          \eapp{\var{sq}}{10} 
        }
      }$
      % 
    \item $\elet{\clam{\enum{}}{x}{
          \clam{\enum{}}{y}{
            \etimes{\var x}{\var y}
          }
        }
      }{f}{
        \eapp
        {\var f}
        {\eapp
          {\enum{4}}
          {\enum{3}}
        }
      }$
      
    \item $\elet{\clam{\enum{}}{x}{
          \eapp{\var f}{\eplus{\var{x}}{\enum{1}}}
        }
      }{f}{
        \eapp{\var{f}}{0}
      }$ 
    \end{itemize}
  }

  \bigskip 




\end{frame}




\begin{frame}{Other examples}
  \label{fr:notype-ef}

  Type check then evaluate the following terms:
  % 
  % 
  \begin{itemize}
  \item {\small $ \eapp{
      \eapp{{\elam{\enum{}}{x}{\elam{\enum{}}{y}{\eplus{\var{x}}{\var{y}}}}}}
      {\enum{1}}
    }{\enum{2}}
    $}
    % 
  \item $\eapp
    {\elam{\tau}{x}{\eapp
        {\var{x}}
        {\var{x}}}
    }
    {\enum{2}}$
    % 
  \item $\eapp
    {\elam{\tau}{x}{\eapp
        {\var{x}}
        {\var{x}}}
    }
    {\elam{\tau'}{x}{\eapp
        {\var{x}}
        {\var{x}}}
    }$
    % 
  \end{itemize}

  \bigskip

  The same programs in concrete syntax:
  \begin{itemize}
  \item  $
    \capp{
      \capp{{\clam{\enum{}}{x}{\clam{\enum{}}{y}{{\var{x}+\var{y}}}}}}
      {\enum{1}}
    }{\enum{2}}
    $
  \item $\capp
    {\clam{\tau}{x}{\capp
        {\var{x}}
        {\var{x}}}
    }
    {\enum{2}}$
    % 
      \item $\capp
    {\clam{\tau}{x}{\capp
        {\var{x}}
        {\var{x}}}
    }
    {\clam{\tau'}{x}{\capp
        {\var{x}}
        {\var{x}}}
    }$
    % 
  \end{itemize}

  % 
  
\end{frame}


\begin{frame}
  \label{fr:eval-strat-ef}
  \frametitle{Evaluation strategies}
  Remember the difference between \emph{call-by-name} vs \emph{call-by-value} evaluations?
  
  \bigskip

  \begin{itemize}
  \item $\eapp{\elam{\enum{}}{x}{\eplus{\var{x}}{\var{x}}}}{e}$
    % $(\lambda \var{x} . \ \var{x} + \var{x}) \ e$
  \item $\eapp{\elam{\enum{}}{x}{\enum{42}}}{e}$
    
  \end{itemize}

  \bigskip

  Which evaluation strategy is the best for these terms?

\end{frame}

%%% Local Variables:
%%% mode: latex
%%% TeX-master: "main"
%%% End:
