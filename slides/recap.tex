\begin{frame}
  \frametitle{Feedback from Week 19}
  \begin{itemize}
  \item Missed lectures: hard to catch up.
  \item How to break down expressions?
  \item Difficulty understanding binders (let)
  \item When to apply plr or pll, and why is there separate rules?
  \item Whether we should write ``$\valjudge{e}$'' judgements explicitly
  \item Syntax of substitutions
  \item Terminology rule names, gamma ($\Gamma$), tau ($\tau$)
  \item Full definitions of all languages will be on Moodle
  \end{itemize}
\end{frame}

\begin{frame}
  \frametitle{Recap}
  Judgements covered so far:
  
  \bigskip
  
  \begin{itemize}
    \setlength\itemsep{1.5em}
  \item \emph{Type system:}
    $\tyjudge{\var{x_1} : \tau_1, \ldots, \var{x_k}:\tau_k}{e}{\tau}$
    means that, assuming that variable $\var{x_i}$ has type $\tau_i$
    ($\forall 1 \leq i \leq k$), term $e$ has type $\tau$.
    % 
  \item \emph{Semantics (1):} $\valjudge{e}$ means that term $e$ is a \emph{value} (it
    cannot be evaluated further).
    % 
  \item \emph{Semantics (2):} $\jtrans{e}{e'}$ means that term $e$
    evaluate to term $e'$ (in one step).
  \end{itemize}
\end{frame}



%%% Local Variables:
%%% mode: latex
%%% TeX-master: "main"
%%% End:
