\section{Polymorphism}

\begin{frame}[fragile,fragile]
  \frametitle{Introduction}
  Take a language such as $\lggeEF$ with base types $\enum{}$ and
  $\estr{}$ and product types.
  
  Can we define \emph{one} function which takes a triple (e.g.,
  $\cpair{a}{\cpair{b}{c}}$) and returns the middle element (i.e., $b$)?

  % \bigskip

  \[
  \elam{\typrod{\enum{}}{\typrod{\enum{}}{\enum{}}}}{x}{ \eprl{\eprr{\var x}}}
  \]
  
  \begin{lstlisting}[language=Haskell]
-- Haskell
middle :: (t1,(t2,t3)) -> t2
middle (x,(y,z)) = y
\end{lstlisting}
  
\begin{lstlisting}[language=Java]
// Java
// (assume class Pair<T, V> exists)
public <S> S getMiddle(Pair<T,Pair<S,V>> triple){
  return triple.getRight().getLeft();
}
\end{lstlisting}

\bigskip
  
  More examples?
  
\end{frame}



\begin{frame}
  \frametitle{Polymorphicity}
  \begin{itemize}
  \item The term \emph{polymorphism} refers to a range of language mechanisms
    that allow a single part of a program to be used with different
    types in different contexts.
    % 
  \item The languages we have considered so far are all
    \emph{monomorphic} in that every expression has at most \emph{one}
    type (given the types of its free variables).
    % 
  \item \emph{Overloading} is a form of ``ad-hoc polymorphism''. It
    associates a single function symbol with many implementations.
    The compiler chooses an appropriate implementation for each
    application of the function, based on the types of the arguments.
  \end{itemize}
\end{frame}

\subsection{Syntax}

\begin{frame}
  \frametitle{Language $\lggeF$}
  \[
  \begin{array}{llclll}
    \TYPES & \tau & \Coloneqq & \tvar{t} & \tvar{t} & \text{type variable}
    \\ 
           &&& \tyarr{\tau_1}{\tau_2}  & \tau_1 \rightarrow \tau_2 & \text{function}
    \\
           &&& \tyall{t}{\tau} & \ctyall{t}{\tau} & \text{polymorphic}
    \\
    \\
    \EXPS & e & \Coloneqq & \var{x} & \var{x} & \text{variable}
    \\
           &&& \elam{\tau}{x}{e} & \clam{\tau}{x}{e} & \text{abstraction}
    \\
           &&& \eapp{e_1}{e_2} & \capp{e_1}{e_2} & \text{application}
    \\
           &&& \eLAM{t}{e} & \cLAM{t}{e} & \text{type abstraction}
    \\
           &&& \eAPP{\tau}{e} & \cAPP{\tau}{e} & \text{type application}
  \end{array}
  \]
\end{frame}


\begin{frame}
  \frametitle{Exercises}
  \begin{itemize}
  \item Define the identify function ($\texttt{id}(x) = x$)
  \item Define a polymorphic function to compose function together ($f \circ g = f(g(x))$)
  \item Any more ``standard'' polymorphic functions?
  \end{itemize}
\end{frame}

% L(T) . \x:t . x
% L(T1) . L(T2) . L(T2) \f:T2->T3 \g:T1->T2 = f(g(x))


\subsection{Statics}

\begin{frame}
  \frametitle{Judgements}

  Two judgement forms and two kinds of environments, the hypotheses in
  $\Delta$ have the form $\tvar{t} \ \typeok$, where $\tvar{t}$ is a
  variable of sort $\TYPES$ and the hypotheses in $\Gamma$ have the
  form $\var{x} : \tau$, where $\var{x}$ is a variable of sort $\EXPS$.

  \bigskip

  \begin{itemize}
  \item   {\Huge
      $    
      {\typejudge{\Delta}{\tau}}
      $   
    }

    says that the type $\tau$ is well-formed, under the hypotheses in $\Delta$

    \bigskip

  \item   {\Huge
      $    
      {\tyFjudge{\Delta}{\Gamma}{e}{\tau}}
      $   
    }
    
    says that the expression $e$ has type $\tau$, under  the hypotheses in $\Delta$ and $\Gamma$
  \end{itemize}
\end{frame}



\note{As usual, we write $\varnothing$ for the empty context $\Delta$ or $\Gamma$}


\begin{frame}
  \frametitle{Well-formed types}
 \[
  \tyrule
  {}
  {\,}
  {\typejudge{\Delta, \tvar{t} \ \typeok , \Delta_2}{\tvar{t}} }
  \]

  
  \[
  \tyrule
  {}
  {
    \typejudge{\Delta}{\tau_1}
    \and
    \typejudge{\Delta}{\tau_2}
  }
  {\typejudge{\Delta}{\tyarr{\tau_1}{\tau_2}} }
  \]

  
  \[
  \tyrule
  {}
  {
    \typejudge{\Delta, \tvar{t} \ \typeok }{\tau}
  }
  {\typejudge{\Delta}{\tyall{t}{\tau}} }
  \]
  
\end{frame}

\begin{frame}
  \frametitle{Examples}
  Are these types well-formed?
  \begin{itemize}
  \item $\tyall{t}{\tyarr{\tvar t}{\tvar t}}$
  \item $\tyall{t_1}{\tyall{t_2}{\tvar{t_1}}}$
  \item $\tyall{t_1}{\tyall{t_2}{\tyarr{\tvar{t_2}}{\tvar{t_1}}}}$
  \end{itemize}
\end{frame}


\begin{frame}
  \frametitle{Typing judgements (1)}
  \[
  \tyrule
  {\ftyrulename{var}}
  {\,}
  {\tyFjudge{\Delta}{\Gamma_1, \var{x}: \tau, \Gamma_2 }{\var{x}}{\tau} }
  \]


  \[
  \tyrule
  {\ftyrulename{lam}}
  {\tyFjudge{\Delta}{\Gamma, \var{x} : \tau_1}{e}{\tau_2}}
  {\tyFjudge{\Delta}{\Gamma}{\elam{\tau_1}{x}{e}}{\tyarr{\tau_1}{\tau_2}}}
  \]

  \[
  \tyrule
  {\ftyrulename{ap}}
  {
    \tyFjudge{\Delta}{\Gamma}{e_1}{\tyarr{\tau_1}{\tau_2}}
    \and
    \tyFjudge{\Delta}{\Gamma}{e_2}{\tau_2}
  }
  {
    \tyFjudge{\Delta}{\Gamma}{\eapp{e_1}{e_2}}{\tau_2}
  }
  \]

\end{frame}


\begin{frame}
  \frametitle{Typing judgements (2)}
  

  \[
  \tyrule
  {\ftyrulename{tlam}}
  {\tyFjudge{\Delta, \tvar{t} \ \typeok}{\Gamma}{e}{\tau}}
  {
    \tyFjudge{\Delta}{\Gamma}{\eLAM{t}{e}}{\tyall{t}{\tau}}
  }
  \]


  \[
  \tyrule
  {\ftyrulename{tap}}
  {
    \typejudge{\Delta}{\tau}
    \and
    \tyFjudge{\Delta}{\Gamma}{e}{\tyall{t}{\tau'}}
  }
  {
    \tyFjudge{\Delta}{\Gamma}{\eAPP{\tau}{e}}{\tsubs{\tau'}{\tau}{t}}
  }
  \]

\end{frame}

\begin{frame}
  \frametitle{Examples}
  Assuming we have product types, do these expression have a type?
  \begin{itemize}
  \item $\eLAM{t}{\elam{\tvar{t}}{x}{\var x}}$
  \item $\eLAM{t}{\elam{\typrod{\tvar{t}}{\typrod{\tvar{t}}{\tvar{t}}}}{x}{ \eprl{\eprr{\var x}}}}$
  \item Check for the solution of $f \circ g$
  \end{itemize}
\end{frame}

% let as lambda => (\id . <id "hello", id 123>) (\x . x)


\subsection{Semantics}

\begin{frame}
  \frametitle{Dynamics}
  \textbf{Call-by-name} semantics
  \[
  \semrule
  {\fsemrulename{lam}}
  {\,}
  {\valjudge{\elam{\tau}{x}{e}}}
  \]

  \[
  \semrule
  {\fsemrulename{tlam}}
  {\,}
  {\valjudge{\eLAM{t}{e}}}
  \]

\end{frame}

\begin{frame}
  \frametitle{Dynamics}
  \[
  \semrule
  {\fsemrulename{ap}}
  {\jtrans{e_1}{e'_1}}
  {
    \jtrans{\eapp{e_1}{e_2}}{\eapp{e'_1}{e_2}}
  }
  \]

  \[
  \semrule
  {\fsemrulename{lav}}
  {\,}
  {\jtrans{\eapp{\elam{\tau}{x}{e_1}}{e_2}}{\subs{e_1}{e_2}{x}}}
  \]
\end{frame}




\begin{frame}
  \frametitle{Dynamics}
  \[
  \semrule
  {\fsemrulename{ap}}
  {\,}
  {
    \jtrans
    {\eAPP{\tau}{\eLAM{t}{e}}}
    {\tsubs{e}{\tau}{t}}
  }
  \]

  \[
  \semrule
  {\fsemrulename{lav}}
  {\jtrans{e}{e'}}
  {
    \jtrans
    {\eAPP{\tau}{e}}
    {\eAPP{\tau}{e'}}
  }
  \]
\end{frame}


\begin{frame}
  \frametitle{Expressivity of $\lggeF$}
  Product and sum type can be expressed with the construct
  of $\lggeF$!
  %
  See Chapter 16.2 of PFPL.
\end{frame}

%%% Local Variables:
%%% mode: latex
%%% TeX-master: "main"
%%% End:
