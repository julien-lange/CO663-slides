\section{Introduction}



\begin{frame}
  \frametitle{Introduction}
  \begin{quotation}
    \Large
    A good language is one people use.
  \end{quotation}

\end{frame}

\begin{frame}
  \frametitle{Language rankings}

  \begin{itemize}[<+->]
  \item \textbf{Popularity:} \url{http://www.tiobe.com/tiobe-index}
    (search engine queries containing the name of the language)
    %
    % 
  \item \textbf{Popularity:} \url{http://pypl.github.io/PYPL.html} (queries for language tutorials on Google)
    %
  \item \textbf{Interest:}
    \url{https://insights.stackoverflow.com/survey/2019\#most-loved-dreaded-and-wanted}
  \item \textbf{Usage:} % \url{https://octoverse.github.com/projects\#languages} 
    \url{https://www.benfrederickson.com/ranking-programming-languages-by-github-users}
    (number of contributors)
    % 
  \item \textbf{Salaries:}
    \url{https://insights.stackoverflow.com/survey/2019\#top-paying-technologies}
  \end{itemize}
  \pause 
  More GitHub data: \url{https://githut.info}
\end{frame}

\begin{frame}
  \frametitle{Some a features of a good language}
  \begin{itemize}
  \item Readability (elegant syntax)
  \item Simplicity (clean, orthogonal constructs)
  \item \textbf{Safety} (programs won't go wrong)
  \item Support for programming in the large (modularity)
  \item Efficiency (good execution model and tools)
  \end{itemize}
\end{frame}


\begin{frame}
  \frametitle{Tensions in PL design}
  Sometimes these goals are conflicting:
  \begin{itemize}
  \item Types provide safety but might restrict expressiveness.
  \item Safety check (e.g., array out of bound checks) might affect
    run-time performances.
  \end{itemize}

  \bigskip
  
  A lot of research in PL is about trying to achieve these goals while
  appeasing these conflicts.
\end{frame}




\begin{frame}
  \frametitle{Introduction}
  \begin{itemize}
  \item Why learn the principles of language design?
    % 
  \item What is the difference between language \emph{design} and \emph{implementation}?
    % 
  \item In what sort of situation would you have to define \emph{your own
    language}?
    % 
  \item What would you have to think of when \emph{designing} your
    own language?
  \end{itemize}
\end{frame}




\begin{frame}
  \frametitle{Example of balance in PL: Rust}
 
  \begin{itemize}
  \item Improve safety guarantees by preventing concurrent threads to
    share memory.
  \item Improve runtime performance by avoiding the need for a garbage
    collector.
  \item All this was largely inspired by research in programming languages.
  \end{itemize}

  \bigskip
  \pause
  What is the cost of these features?
  
\end{frame}


\begin{frame}
  \frametitle{Example: modernisation of languages}
  \begin{itemize}[<+->]
  \item \textbf{\href{https://www.typescriptlang.org/}{TypeScript}}: superset
    of JavaScript with (optional) type annotation --- started at Microsoft.
    % 
  \item \textbf{\href{https://hacklang.org/}{Hack}}: dialect of PHP with type
    annotation (gradual types) --- started at Facebook.
    %
  \item WhatsApp/Facebook have similar plans for Erlang \ldots
  \end{itemize}
\end{frame}

\begin{frame}
  \frametitle{Programming language design: examples}

  \begin{itemize}
  \item \href{https://dl.acm.org/citation.cfm?id=2984004}{Java and
      scala's type systems are unsound: the existential crisis of null
      pointers} (Nada Amin and Ross Tate).
    % 
  \item \href{https://dl.acm.org/citation.cfm?id=3009871}{Java generics are turing complete} (Radu Grigore).
    % 
  \item The Go community is currently attempting to add
    \textbf{generics} in Go. See the
    \href{https://go.googlesource.com/proposal/+/master/design/go2draft-generics-overview.md}{description of the problem}
    and the
    \href{https://go.googlesource.com/proposal/+/master/design/go2draft-contracts.md}{draft
      design}.
  \end{itemize}
\end{frame}



\begin{frame}% [allowframebreaks]
  \frametitle{Programming language (PL): definitions}
  
  \begin{itemize}
  \item A PL is a system of notation for describing computations. A
    useful PL must therefore be suited for both
    \emph{describing} (i.e., for human writers and readers of
    programs), and for \emph{computation} (i.e., for efficient
    implementation on computers). [R.D. Tennent, 1985]

  \item PLs \textbf{express computations} in a form comprehensible to both
    people and machines. [R. Harper, 2014]

  \item A PL is a formal language that specifies a set of instructions
    that can be used to produce various kinds of output. PLs generally
    consist of instructions for a computer. PLs can be used to create
    programs that implement specific algorithms. [Wikipedia, 2019]
  \end{itemize}  
\end{frame}


\begin{frame}
  \frametitle{Syntax, types, and semantics}
  \begin{aquote}{R. Harper}
    \emph{Types} are the central organizing principle of the theory of
    programming languages.
    % 
    Language features are manifestations of \emph{type structure}. The
    \emph{syntax} of a language is governed by the constructs that
    define its types, and its \emph{semantics} is determined by the
    interactions among those constructs. The \emph{soundness of a
      language} design (the absence of ill-defined programs) follows
    naturally.
  \end{aquote}
\end{frame}


\begin{frame}
  \frametitle{Syntax, types, and semantics}
  A programming language is described by
  \begin{itemize}
  \item its \emph{syntax} of programs (what is a term of the language),
    % 
  \item its \emph{semantics}, i.e., the meaning of programs, how
    programs are evaluated, how they behave,
    % 
  \item its \emph{type system}, i.e., the (context sensitive)
    constraints on the formation of program phrases.
  \end{itemize}

  \bigskip Where do you find the semantics of your favourite programming
  language? How is it defined?
\end{frame}








% \note{
%   \frametitle{Goals}
%   \begin{itemize}
%   \item Knowledge: what is a programming language, what is type safety, semantics, type systems, etc
%   \item Do
%     \begin{itemize}
%     \item Augment a simple language with simple language constructs
%       (e.g., if then else, new data types) and preserve type safety
%     \item Apply capture avoiding substitution
%     \item Given a set of semantics rules, be able to show the reduction
%     \end{itemize}
%   \end{itemize}
% }


% \begin{frame}
%   \frametitle{Objectives of Part II}
%   \begin{itemize}
%   \item Techniques for modeling programs/languages mathematically
%   \item To learn a comprehensive framework for formulating and
%     analysing a broad range of ideas in PLs
%     % 
%   \item Language features are manifestations of type structures
%     % 
%   \end{itemize}
% \end{frame}


% \note{
% %\begin{frame}
%   \frametitle{Languages}
%   \begin{itemize}
%   \item Java (OO?)
%   \item Rust (affine?)
%   \item Haskell (start? HO function)
%   \item Python (dynamic typing)
%   \item OCaml (mix of Java and Haskell?)
%   \end{itemize}
%   Maybe spent one lecture explaining proof by induction and a couple
%   of examples? (possibly state to be non-examinable)
% %\end{frame}
% }



\begin{frame}
  \frametitle{Main textbook}

  This part of the course follows loosely, ``\emph{Practical Foundations for
  Programming Languages}'' [PFPL] (Second Edition) by Robert Harper.

  \bigskip

  \begin{tabular}[t]{cc}
    \includegraphics[width=0.4\textwidth]{harper.jpg}
    &
     \begin{tabular}[t]{l}
       Available in the library.
     \end{tabular} 
  \end{tabular}
\end{frame}


\begin{frame}
  \frametitle{Other references}
  Other useful references:
  \begin{itemize}
  \item \emph{Types and Programming Languages} [TAPL], by Benjamin C. Pierce
    (available in the library)
    % 
  \item \emph{Programming Languages: Application and Interpretation},
    by Shriram Krishnamurthi (available
    \href{http://cs.brown.edu/~sk/Publications/Books/ProgLangs/}{online})
    % 
  \item \emph{Advanced Topics in Types and Programming Languages} [ATTAPL],
    edited by Benjamin C. Pierce (available in the library)
    % 
  \item \emph{Introduction to Lambda Calculus}, by Henk Barendregt and Erik Barendsen
    (available \href{http://www.nyu.edu/projects/barker/Lambda/barendregt.94.pdf}{online})
\end{itemize}
\end{frame}


\begin{frame}
  \frametitle{A selection of PL talks}
  \begin{itemize}
  \item \href{https://www.youtube.com/watch?v=06x8Wf2r2Mc}{A History
      of Haskell: being lazy with class} by Simon Peyton Jones (Microsoft/Haskell)
  \item \href{https://www.youtube.com/watch?v=5kj5ApnhPAE}{Public Static Void} by Rob Pike (Google/Go)
  \item \href{https://www.youtube.com/watch?v=O5vzLKg7y-k}{The Rust Programming Language} by Aaron Turon (Mozilla/Rust)
  \item \href{https://www.youtube.com/watch?v=60nXRNjEeo4}{The
      Challenges of Creating a Programming Language} Martin Odersky
    (EPFL/Scala)
  \item \href{https://www.youtube.com/watch?v=9lWrt6H6UdE}{What's Different In Dotty} by Martin Odersky (EPFL/Scala)
  \item \href{https://www.youtube.com/watch?v=xc72SYVU2QY}{Move Fast to Fix More Things} by Peter O'Hearn (UCL/Facebook)
  \item \href{https://www.youtube.com/watch?v=RoqDQgzwO00}{Gradual Typing (Lecture 1 from OPLSS'17)} by Ron Garcia (UBC)
  \item \href{https://www.youtube.com/watch?v=8Xyk_dGcAwk}{The Type Soundness Theorem That You Really Want to Prove (and now you can)} by Derek Dreyer
  \end{itemize}
\end{frame}




\begin{frame}
  \frametitle{Assignment \#2: paper presentations}
  
  \begin{enumerate}
  \item Choose \emph{one} paper from list on the
    \href{https://rgrig.github.io/plad/essay.html}{course website} and
    form a group of three. Use \href{https://piazza.com}{piazza}!

  \item We will allow at most \textbf{one group per paper}.
    % \begin{enumerate}
    % \item \href{https://www.cis.upenn.edu/~bcpierce/papers/fj-toplas.pdf}{Featherweight Java}
    % \item \href{https://dl.acm.org/citation.cfm?id=2951945}{Disjoint intersection types}
    % \item \href{http://ecee.colorado.edu/~siek/pubs/pubs/2006/siek06:_gradual.pdf}{Gradual Typing for Functional Languages}
    % \item \href{https://lirias.kuleuven.be/bitstream/123456789/203434/1/ownership.ps}{Ownership types for flexible alias protection}
    % \item \href{https://dl.acm.org/citation.cfm?id=113468}{Refinement types for ML}
    % \item \href{https://www.cs.cornell.edu/andru/papers/popl99/popl99.pdf}{JFlow: Practical Mostly-Static Information Flow Control}
 
    % \item \href{http://homepages.inf.ed.ac.uk/wadler/papers/need-journal/need-journal.ps}{The Call-by-Need Lambda Calculus}
      

    % \item \href{https://doi.org/10.1109/LICS.1993.287570}{Typing and Subtyping for Mobile Processes}
    % \end{enumerate}
  \item Announce which paper you've chosen (to me
    \href{mailto:j.s.lange@kent.ac.uk}{j.s.lange@kent.ac.uk})
    \emph{before} Friday Week 20, 4pm.
    % 
  \item Read and understand the paper, then \emph{prepare a presentation} for it
    (cover at least types, semantics, and some results)
    % 
  \item \emph{Present} the paper in groups in Weeks 23 \& 24.
    % 
  \item Submit your slides via Moodle \textbf{before} your presentation.
    % 
    % \item Submit your supporting material (notes, slides, etc) by Friday Week 22.
  % \item \emph{Present} the paper. Bring your supporting
  %   material.
  \end{enumerate}
\end{frame}



\begin{frame}
  \frametitle{Remarks re:\ paper reading}
  \begin{itemize}
  \item Some tips:
    \begin{itemize}
    \item \href{https://blog.acolyer.org/2018/01/26/a-practitioners-guide-to-reading-programming-languages-papers/}{The Morning Paper}
    \end{itemize}
  \item What is \emph{more important}:
    \begin{itemize}
    \item To be able to summarise the overall contribution of the paper
    \item To understand semantics and type systems
    \item To be able to explain the main result of the approach (pros and cons)
    \end{itemize}
  \item What is \emph{less important}:
    \begin{itemize}
    \item Algorithmic aspects
    \item Proofs
    \end{itemize}
  \item Short paper $\neq$ easy paper. If you choose a long
    paper, you can decide to focus your summary/presentation on part
    of it.
  \item Feel free to use additional material (other papers, tutorials, lectures, etc).
  \end{itemize}

  \bigskip

  \emph{Ask if in doubt!}
\end{frame}



%%% Local Variables:
%%% mode: latex
%%% TeX-master: "main"
%%% End:
