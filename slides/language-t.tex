\section{Recursion}


% \begin{frame}
%   \frametitle{Housekeeping} 
%   \begin{itemize}[<+->]
%   \item Explain: ``if $\tyjudge{}{e}{\tau}$ and $\jtrans{e}{e'}$, then $\tyjudge{}{e'}{\tau}$''.
%     \begin{itemize}
%     \item If $e$ is well-typed and $e$ reduces to $e'$, then $e'$ is
%       also well-typed (and both $e$ and $e'$ have the same type).
%     \item This statement is also called \emph{type preservation} (cf.\ type safety,  slide~\ref{fr:ty-safety}).
%     \end{itemize}
%   \item Whiteboard pictures on Moodle $\implies$ ask on Piazza.
%   \item \LaTeX\ in the lectures $\implies$ too risky! 
%   \end{itemize}
% \end{frame}

\begin{frame}
  \frametitle{Introduction}

  \begin{center}
    {\large
      What is the \emph{expressive power} of the languages we have studied so far?
    }
  \end{center}

  \pause

  \bigskip  

  \begin{itemize}[<+->]
  \item We can specify the expressivity of a programming language by
    considering the set of computable functions it can represent.
    % 
  \item Most programming languages are \emph{universal}, i.e., Turing
    complete (meaning that they can be used to simulate a Turing
    machine).
    % 
  \item Hence, expressivity of programming languages is generally
    concerned with questions such as: can construct $C$ in language
    $L$ be simulated in language $L'$?
    % 
  \item For more on expressivity of programming languages beyond
    computational power, see, e.g.,
    \emph{``\href{https://www.sciencedirect.com/science/article/pii/016764239190036W}{On
        the expressive power of programming languages}''}, by Matthias
    Felleisen.
  \end{itemize} 

\end{frame}


% \note{In particular, what sort of recursive programs (or programs that
%   ``loop'' infinitely) can we write in the languages we have seen thus far?}

\begin{frame}
  \frametitle{Language \lggeT}
  % 
  \[
  \begin{array}{llclll}
    \TYPES & \tau & \Coloneqq & \tynat  & \tynat & \text{natural}
    \\
           &&& \tyarr{\tau_1}{\tau_2}  & \tau_1 \rightarrow \tau_2 & \text{function}
    \\
    \\ \pause
    \EXPS & e & \Coloneqq & \var{x} & \var{x} & \text{variable}
    \\
           &&& \ez &  \ez & \text{zero}
    \\
           &&& \esucc{e} &  \esucc{e} & \text{successor}
    \\
           &&& \elam{\tau}{x}{e} & \clam{\tau}{x}{e} & \text{abstraction}
    \\
           &&& \eapp{e_1}{e_2} & \capp{e_1}{e_2} & \text{application}
    \\
           &&& \eiter{e_0}{x}{e_1}{e} & \multicolumn{2}{c}{\citer{e_0}{x}{e_1}{e}}                                      
  \end{array}
  \]

  \pause

  We write $\bnum{3}$ for $\esucc{\esucc{\esucc{\ez}}}$.
\end{frame}

\begin{frame}
  \frametitle{Statics}
  % (same as $\lggeEF$)
  \[
  \tyrule
  {\ttyrulename{var}}
  {\,}
  {\tyjudge{\Gamma_1, \var{x}: \tau, \Gamma_2}{\var{x}}{\tau} }
  \]


  \pause 
  \bigskip

  \[
  \tyrule
  {\ttyrulename{nat}}
  {\,}
  {\tyjudge{\Gamma}{\ez}{\tynat}}
  \qquad\qquad\qquad
  \tyrule
  {\ttyrulename{num}}
  {\tyjudge{\Gamma}{{e}}{\tynat{}}}
  {\tyjudge{\Gamma}{\esucc{e}}{\tynat{}}}
  \]
\end{frame}




\begin{frame}
  \frametitle{Statics}
  
  \[
  \tyrule
  {\ttyrulename{lam}}
  {\tyjudge{\Gamma, \var{x} : \tau_1}{e}{\tau_2}}
  {\tyjudge{\Gamma}{\elam{\tau_1}{x}{e}}{\tyarr{\tau_1}{\tau_2}}}
  \]

  \pause

  \[
  \tyrule
  {\ttyrulename{ap}}
  {
    \tyjudge{\Gamma}{e_1}{\tyarr{\tau_1}{\tau}}
    \and
    \tyjudge{\Gamma}{e_2}{\tau_1}
  }
  {
    \tyjudge{\Gamma}{\eapp{e_1}{e_2}}{\tau}
  }
  \]


  \pause

  \bigskip

  \bigskip

  NB: this rules are the same as for Language $\lggeEF$.
\end{frame}



\begin{frame}
  \frametitle{Statics}
  \[
  \tyrule
  {\ttyrulename{ite}}
  {
    \tyjudge{\Gamma}{e}{\tynat}
    \and
    \tyjudge{\Gamma}{e_0}{\tau}
    \and
    \tyjudge{\Gamma, \var{x} : \tau}{e_1}{\tau}
  }
  {
    \tyjudge{\Gamma}{\eiter{e_0}{x}{e_1}{e}}{\tau}
  }
  \]

  \bigskip

  \pause
  
  Why do $e_0$ and $e_1$ need to have the same type?
  
\end{frame}


\begin{frame}
  \frametitle{Dynamics}
  \emph{Eager} semantics:
  \[
  \semrule{\tsemrulename{z}}
  {\,}
  {\valjudge{\ez}}
  \qquad \qquad  \qquad
  \semrule{\tsemrulename{s}}
  {\valjudge{e}}
  {\valjudge{\esucc{e}}}
  \]

  \[
  \semrule
  {\tsemrulename{lam}}
  {\,}
  {\valjudge{\elam{\tau}{x}{e}}}
  \]


\end{frame}


\begin{frame}
  \frametitle{Dynamics}
  
  % (same as $\lggeEF$)

  \[
  \semrule
  {\tsemrulename{ss}}
  {\jtrans{e}{e'}}
  {
    \jtrans{\esucc{e}}{\esucc{e'}}
  }
  \]

  \[
  \semrule
  {\tsemrulename{ap}}
  {\jtrans{e_1}{e'_1}}
  {
    \jtrans{\eapp{e_1}{e_2}}{\eapp{e'_1}{e_2}}
  }
  \]
\end{frame}

\begin{frame}
  \frametitle{Dynamics}
  %  (same as $\lggeEF$)

  \[
  \semrule
  {\tsemrulename{lan}}
  {\valjudge{e_1} 
    \and \jtrans{e_2}{e'_2}
  }
  {\jtrans{\eapp{e_1}{e_2}}{\eapp{e_1}{e'_2}}}
  \]

  \[
  \semrule
  {\tsemrulename{lav}}
  {\valjudge{e_2}}
  {\jtrans{\eapp{\elam{\tau}{x}{e_1}}{e_2}}{\subs{e_1}{e_2}{x}}}
  \]
\end{frame}

\begin{frame}
  \frametitle{Dynamics (iterator)}
  

  \begin{itemize}[<+->]
  \item Evaluate the parameter
    \[
    \semrule
    {\tsemrulename{rin}}
    {\jtrans{e}{e'}}
    {
      \jtrans{\eiter{e_0}{x}{e_1}{e}}{\eiter{e_0}{x}{e_1}{e'}}
    }
    \]
    % 
  \item Case if zero 
    \[
    \semrule
    {\tsemrulename{r0}}
    {\,}
    {
      \jtrans{\eiter{e_0}{x}{e_1}{\ez}}{e_0}
    }
    \]
  \item Case if strictly positive
    \[
    \semrule
    {\tsemrulename{rs}}
    {\valjudge{{e}}}
    {
      \jtrans{\eiter{e_0}{x}{e_1}{\esucc{e}}}{\subs{e_1}{{\eiter{e_0}{x}{e_1}{{e}}}}{x}}
    }
    \]
  \end{itemize}


  NB: $\eiter{e_0}{x}{e_1}{e}$ stands for ${\citer{e_0}{x}{e_1}{e}}$     
\end{frame}

\begin{frame}
  \frametitle{Alpha equivalence ($\aeq$)}
  \begin{itemize}[<+->]
  \item Two programs are \emph{$\alpha$-equivalent} if they are
    identical up to the choice of \emph{bound} variables.
    % 
    \pause
    \[
    \elam{\tynat}{x}{\esucc{\var{x}}} \aeq   \elam{\tynat}{y}{\esucc{\var{y}}}
    \]

    \pause


    \[
    \esucc{\var{x}} \naeq \esucc{\var{y}}
    \]
    
  \end{itemize}
\end{frame}

\begin{frame}
  \frametitle{Alpha equivalence ($\aeq$)}
  \begin{itemize}
  \item \emph{Bound} variables can be renamed \emph{consistently}
    without changing the meaning of a program.
    \[
    \begin{array}{c}
      \elam{\tyarr{\tynat}{\tynat}}{x_1}{
      \elam{\tynat}{x_2}{
      \eapp
      {\var{x_1}}
      {\var{x_2}}
      }
      }
      \\
      \pause 
      \qquad
      \aeq
      \elam{\tyarr{\tynat}{\tynat}}{y_1}{
      \elam{\tynat}{y_2}{
      \eapp
      {\var{y_1}}
      {\var{y_2}}
      }
      }
      \\
      \pause
      \qquad
      \naeq
      \elam{\tyarr{\tynat}{\tynat}}{y_1}{
      \elam{\tynat}{y_2}{
      \eapp
      {\underline{\var{y_2}}}
      {\underline{\var{y_1}}}
      }
      }
    \end{array}
    \]
    \pause
  \item It's helpful to rename bound variables so that they are all
    \emph{distinct}, e.g.,
    \[
    \begin{array}{c}
      \eapp
      {\elam{\tau}{x}{{\var{x}}}}
      {\elam{\tau'}{x}{\esucc{\var{x}}}}
      \\ \qquad
      \pause
      \aeq
      \eapp
      {\elam{\tau}{x}{{\var{x}}}}
      {\elam{\tau'}{y}{\esucc{\var{y}}}}
    \end{array}
    \]
    
  \end{itemize}
\end{frame}


\begin{frame}
  \frametitle{Substitution}
  {\Large
    \[
    \subs{e_1}{e_2}{x}
    \]
  }
  \pause
  \begin{itemize}[<+->]
  \item Substitution is the process of ``plugging in'' an object
    ($e_2$) for the variable ($\var{x}$) in a program ($e_1$).
    % 
  \item Only the \emph{free} occurrences of variable $\var{x}$ should
    be replaced!
  \item Example: if
    \[ 
    e_1 = \eapp{\elam{\tynat}{x}{\esucc{\var{x}}}}{\underline{\var{x}}}
    \]
    \pause
    then 
    \[
    \subs{e_1}{\ez}{x} = 
    \pause
    \eapp{\elam{\tynat}{x}{\esucc{\var{x}}}}{\ez}
    \]
  \end{itemize}
  % 
  \pause
  NB: $ \eapp{\elam{\tynat}{x}{\esucc{\var{x}}}}{\underline{\var{x}}} 
  \aeq  \eapp{\elam{\tynat}{y}{\esucc{\var{y}}}}{\underline{\var{x}}} $
\end{frame}


\begin{frame}{Exercises}
  \label{fr:define-t}
  % 
  In $\lggeT$, implement the following functions:
  \begin{itemize}[<+->]
  \item Successor ($\tynat{} \rightarrow \tynat$): $\text{succ} \defi \clam{\tynat{}}{x}{{\color{red} ??}}$
  \item Doubling ($\tynat{} \rightarrow \tynat$): {\color{red} ??}
  \item Addition ($\tynat{} \rightarrow \tynat \rightarrow \tynat$): 
    $\text{add} \defi \clam{\tynat{}}{x}{\clam{\tynat{}}{y}{ {\color{red} ??} }}$
  \item Multiplication ($\tynat{} \rightarrow \tynat \rightarrow \tynat$): {\color{red} ??}
    % 
    % 
  \end{itemize}
   
  % \pause
  % \bigskip 
  
  % Using complex data types (see further chapters):
  % \begin{itemize}
  % \item Predecessor ($\tynat{} \rightarrow \tynat$): {\color{red} ??} (add product types to $\lggeT$)
  % \item Subtraction ($\tynat{} \rightarrow \tynat \rightarrow \tynat$): {\color{red} ??}
  % \end{itemize}
\end{frame}

% Add: \x \y . iter x { z -> y | v -> s(v)}
% Doubling: \x . iter x {z -> z | v -> s(s(v))}
% Multiply: \x \y . iter x {z -> z | v -> add(y,v)}
% Predecessor: \x . pl(iter x {z -> <z,z> | v -> <pr . v, s(pr . v)>})
% Subtraction: \x \y . iter x {z -> y | v -> pred(v)}




% \begin{frame}
%   \frametitle{Housekeeping}
%   \begin{itemize}[<+->]
%   \item What does the following term evaluate to
%     \[
%     \eapp{\elam{\tynat}{x}{\var{x}}}{\esucc{{\var{x}}}}
%     \]
%   \item Is language $\lggeT$ using Peano numbers? Yes, $\lggeT$ is
%     based on Godel's system T (1958), which includes Peano
%     axioms/numbers.
%   \end{itemize}
% \end{frame}
\begin{frame}
  \frametitle{Expressivity of $\lggeT$}
  \begin{itemize}
  \item Language $\lggeT$ provides a mechanism called \emph{primitive
      recursion} (which we used to define arithmetic operations).
    % 
  \item Primitive recursion can characterise the \emph{inductive}
    nature of the natural numbers.
    % 
  \item In $\lggeT$, we can only define \emph{total recursive
      functions} on the natural numbers (functions that \emph{always} return
    something).
    %  
  \item This inductive natures implies that every program comes with a
    proof of its \emph{termination} (we always ``bottom out'' using the
    induction principle).
    % 
  \item To obtain Turing completeness, we need a \emph{partial
      recursive} language (where some programs may not terminate).
  \end{itemize}
\end{frame}



% \separator{Examinable material for 2018 \emph{ends} here.}


\begin{frame}
  \frametitle{Y-Combinator}
  An option to express \emph{general recursion}\footnote{i.e.,
    allowing for partial functions.} is to introduce the $Y$
  combinator.
  
  \bigskip
  Its type is
  \[
  Y : (\tau \rightarrow \tau) \rightarrow \tau
  \]
  

  \pause

  Additional semantic rule:
  \[
  \semrule
  {\tsemrulename{y}}
  {\,}
  {
    \jtrans{Y f}{ f(Y f)}
  }
  \]

  \pause
  \bigskip
 
  NB: in the \emph{untyped} lambda-calculus, it can be expressed as
  \[
  Y \defi \lambda f  .\ (\lambda x . f (x \ x))  \ (\lambda x . f (x \ x))
  \]

  Exercise: try to find a type for that term!
\end{frame}

\separator{Examinable material for 2020 ends here.}

%%% Local Variables:
%%% mode: latex
%%% TeX-master: "main"
%%% End:

%  LocalWords:  computable expressivity Examinable
